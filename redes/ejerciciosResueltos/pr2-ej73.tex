\documentclass[a4paper,10pt]{article}

\usepackage[spanish, activeacute]{babel}
\usepackage{a4wide}
\usepackage{enumerate}
\usepackage[utf8]{inputenc}
%\usepackage{graphicx}
\usepackage{multicol}
\usepackage{multirow}
\usepackage{latexsym}
\usepackage[dvips]{graphicx}
\usepackage{color}
\usepackage{colortbl}



\addtolength\topmargin{-1cm}
\setlength\voffset{0cm}
\setlength\headheight{0cm}
\setlength\headsep{0cm}
\addtolength\textheight{4cm}

\begin{document}

\section*{Ejercicio 73}
   Una empresa necesita un protocolo de nivel de enlace orientado a conexión, full-dúplex y confiable que 
   permita ser implementado sobre un canal con una tasa de ruido del 0,01\% para brindar servicio a varios 
   protocolos de nivel de red. Se pide describir el formato de las tramas de datos y control detallando 
   cada campo y las características del funcionamiento del protocolo para que cumpla con lo especificado.

\section*{Nuestra idea}

La idea es que entre dos hosts se crea una conexión para cada protocolo de la capa superior.  Cada host receptor guarda una tabla de 
\begin{tabular}{|c|c|c|}\hline
Id cx &host origen &Id protocolo\\\hline
\end{tabular}.

El único frame de longitud variable es el de datos y por eso es el único que lleva el campo longitud.

\subsection*{Frame de datos}

\begin{tabular}{|c|c|c|c|c|c|c|c|}\hline
	DST & ORI & TIPO FRAME & ID CX & LEN & SEQN & DATOS & CRC \\ \hline
\end{tabular}

\subsection*{Frames de control}

\begin{description}
\item[Pedir Conexión (Emisor)]\ \\
\begin{tabular}{|c|c|c|c|c|c|}\hline
	DST & ORI & TIPO FRAME & ID CX & ID PROT & CRC\\\hline
\end{tabular}

\item[Tomá Conexión (Receptor)]\ \\
\begin{tabular}{|c|c|c|c|c|}\hline
	DST & ORI & TIPO FRAME & ID CX & CRC\\\hline
\end{tabular}

\item[Cerrar Conexión (Emisor)]\ \\
\begin{tabular}{|c|c|c|c|c|}\hline
	 DST & ORI & TIPO FRAME & ID CX & CRC \\\hline
\end{tabular}

\item[ACK, NAK (Receptor)]\ \\
\begin{tabular}{|c|c|c|c|c|c|}\hline
	DST & ORI & TIPO FRAME & ID CX & SEQN & CRC\\\hline
\end{tabular}

\item[Overload (Receptor)]\ \\
\begin{tabular}{|c|c|c|c|c|}\hline
	DST & ORI & TIPO FRAME & ID CX & CRC\\\hline
\end{tabular}

\end{description}

El Overload podría no tener el id de conexión considerando que ese id viaja para identificar al protocolo y un overload es para todos los protocolos.

\subsection*{Dudas}
\begin{itemize}
	\item ¿Puedo no tener tamaño mínimo de frame? 
	\item ¿Cómo debería fundamentar esa decisión si hay que fundamentarla?
	\item ¿Deberíamos escribir la tabla de ``tipo de frame / largo de frame''? 
\end{itemize}

\end{document}
