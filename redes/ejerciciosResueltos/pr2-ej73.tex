La idea es que entre dos hosts se crea una conexión para cada protocolo de la capa superior.  Cada host receptor guarda una tabla de Id cx & host origen & Id protocolo.

Frame de datos:
\begin{tabular}
	| & DST & ORI & TIPO FRAME & ID CX & LEN & SEQN & DATOS & CRC & |

Frames de control:

Pedir Conexión (Emisor)
\begin{tabular}
	| & DST & ORI & TIPO FRAME & ID CX & ID PROT & CRC & |

Tomá Conexión (Receptor)
\begin{tabular}
	| & DST & ORI & TIPO FRAME & ID CX & CRC & |

Cerrar Conexión (Emisor)
\begin{tabular}
	| & DST & ORI & TIPO FRAME & ID CX & CRC & |

ACK, NAK (Receptor)
\begin{tabular}
	| & DST & ORI & TIPO FRAME & ID CX & SEQN & CRC & |

Overload (Receptor)
\begin{tabular}
	| & DST & ORI & TIPO FRAME & ID CX & CRC & |


El Overload podría no tener el id de conexión considerando que ese id viaja para identificar al protocolo y un overload es para todos los protocolos.

¿Puedo no tener tamaño mínimo de frame? ¿Cómo debería fundamentar esa decisión si hay que fundamentarla?

