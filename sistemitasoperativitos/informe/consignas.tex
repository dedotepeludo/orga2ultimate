\section{Consignas}

Antes de empezar, ejecute:

\texttt{sudo apt-get install man-db manpages manpages-dev}

De esta manera tendr'a acceso a ayuda en l'inea ejecutando:

\texttt{man <comando>}

Por ejemplo:

\texttt{man cp}

Puede adem'as instalar la versi'on en castellano de la ayuda ejecutando:

\texttt{sudo apt-get install manpages-es}

Para acceder a la ayuda en castellano ejecute por ejemplo:

\texttt{man -L es cp}

Tenga presente que no todos los comandos poseen ayuda en castellano.

\texttt{sudo} permite a usuarios normales ejecutar comandos que requieren permisos de administrador.
Al ejecutar un comando con \texttt{sudo} el sistema le pedir'a su password, y no el password del administrador
(llamado \texttt{root} en Linux, siguiendo la tradici'on de Unix). Esto sucede ya que el sistema permite que
ciertos usuarios (que deber'ian corresponderse con usuarios ``privilegiados'' del sistema) puedan utilizar \texttt{sudo}
ingresando solamente su propio password. El usuario por defecto creado en una instalaci'on de Ubuntu tiene este permiso
y por lo tanto en Ubuntu no es necesario una cuenta de administrador o \texttt{root}.

\texttt{apt-get} es el manejador de paquetes de la distribuci'on Ubuntu. Permite instalar, actualizar y desinstalar
programas. M'as adelante lo utilizaremos para instalar las herramientas necesarias para compilar programas en
Linux.

Si se encontrara detr'as de un proxy, antes de utilizar \texttt{apt-get} debe configurar el proxy. Ejecute el siguiente
comando:

\small
\texttt{sudo echo ''Acquire::http::Proxy $\backslash$''http://proxy.uba.ar:8080$\backslash$'';'' > /etc/apt/apt.conf}
\normalsize

\subsection{Comandos b\'asicos de Unix}

\begin{enumerate}

\item \texttt{pwd} Indique qu'e directorio pasa a ser su directorio actual si ejecuta:

\begin{enumerate}
\item \texttt{cd /usr/bin}
\item \texttt{cd}
\item ?`C'omo explica el punto anterior?
\end{enumerate}

\item \texttt{cat} ?`Cu'al es el contenido del archivo \texttt{/home/<usuario>/.profile}?

\item \texttt{find} Liste \textbf{todos} los archivos que comienzan con \texttt{vmlinuz}.

Estos archivos son im'agenes del kernel Linux.

\item \texttt{mkdir} Genere un directorio \texttt{/home/<usuario>/tp}.

\item \texttt{cp} Copie el archivo \texttt{/etc/passwd} al directorio \texttt{/home/<usuario>/tp}.

\item \texttt{chgrp} Cambie el grupo del archivo \texttt{/home/<usuario>/tp/passwd} para que sea el suyo.

\item \texttt{chown} Cambie el due'no del archivo \texttt{/home/<usuario>/tp/passwd} para que sea su usuario.

\item \texttt{chmod} Cambie los permisos del archivo \texttt{/home/<usuario>/tp/passwd} para que:

\begin{itemize}
\item el propietario tenga permisos de lectura, escritura y ejecuci'on
\item el grupo tenga s'olo permisos de lectura y ejecuci'on
\item el resto tenga s'olo permisos de ejecuci'on
\end{itemize}

\item \texttt{grep}

Muestre las l'ineas que tienen el texto ``localhost'' en el archivo \texttt{/etc/hosts}.

Muestre todas las l'ineas que tengan el texto ``POSIX'' de todos los archivos (incluyendo subdirectorios) en \texttt{/etc}.
Evite los archivos binarios y aquellos archivos y directorios que no tienen permiso de lectura para su usuario.

\item \texttt{passwd} Cambie su password.

\item \texttt{rm} Borre el archivo \texttt{/home/<usuario>/tp/passwd}

\item \texttt{ln}

Enlazar el archivo \texttt{/etc/passwd} a los archivos \texttt{/tmp/contra1} y \texttt{/tmp/contra2}.

Hacer un \texttt{ls -l} para ver cuantos enlaces tiene \texttt{/etc/passwd}.

Estos enlaces se llaman ``hardlinks''. Cada nuevo enlace referencia el mismo espacio ocupado del disco r'igido,
y por lo tanto cada hardlink es igual de representativo de esos bytes ocupados del disco r'igido.
El espacio ocupado solamente se liberar'a cuando todos los enlaces hayan sido borrados.

Ahora enlace el archivo \texttt{/etc/passwd} de manera ``soft'' al archivo \texttt{contra3}.

Verifique con \texttt{ls -l} que no aument'o la cantidad de enlaces de \texttt{/etc/passwd}.

Estos enlaces se llaman ``softlinks'' y apuntan no a los bytes del disco r'igido sino a la ruta del archivo a ser enlazado.
Operar sobre el softlink es igual que operar sobre el archivo, sin embargo los softlinks no cuentan en la cantidad de
enlaces (ya que no apuntan a los bytes ocupados del disco r'igido) y pueden ser borrados sin afectar al archivo original,
aunque si se borra el archivo original el softlink quedar'a hu'erfano y no apuntar'a a nada.

\item \texttt{mount}

Monte el CD-ROM de instalaci'on de Ubuntu JeOS y liste su contenido.

Para hacer esto deber'a especificar la ISO de instalaci'on de Ubuntu JeOS como CD-ROM de la m'aquina virtual.
Si bien puede hacer esto como lo hizo para instalar el sistema, si la m'aquina virtual est'a corriendo debe hacer click
derecho en el 'icono con forma de CD-ROM en la esquina inferior derecha de la m'aquina virtual, y seleccionar
\textbf{CD/DVD-ROM Image...} (ver figura \ref{isocdrom3}). En la ventana que aparece seleccione la ISO de instalaci'on
(ver figura \ref{isocdrom4}).

%\Imagen{vb_ubuntu/vb_isocdrom3.png}{7cm}{Usando una imagen ISO con la m'aquina virtual corriendo.}{isocdrom3}
%\Imagen{vb_ubuntu/vb_isocdrom4.png}{7cm}{Seleccionando la ISO de instalaci'on.}{isocdrom4}

Presente los filesystems que tiene montados.

\item \texttt{df} ?`Qu'e espacio libre tiene cada uno de los filesystems montados?

\item \texttt{ps} ?`Cu'antos procesos de usuario tiene ejecutando? Indique cu'antos son del sistema.

\item \texttt{umount} Desmonte el CD-ROM de instalaci'on de Ubuntu JeOS.

\item \texttt{uptime} ?`Cuanto tiempo lleva ejecutando su m'aquina virtual?

\item \texttt{uname} ?`Qu'e versi'on del kernel de Linux est'a utilizando?

\end{enumerate}

\subsection{Salida est\'andar y pipes}

\begin{enumerate}

\item STDOUT

\begin{enumerate}

\item Conserve en el archivo \texttt{/home/<usuario>/tp/config} la salida del comando \texttt{ls} que muestra todos los
archivos del directorio \texttt{/etc} y de los subdirectorios bajo \texttt{/etc}.

\item Presente cuantas l'ineas, palabras y caracteres tiene \texttt{/home/<usuario>/tp/config}.

\item Agregue el contenido, ordenado alfab'eticamente, del archivo \texttt{/etc/passwd} al final del
archivo \texttt{/home/<usuario>/tp/config}.

\item Presente cuantas l'ineas, palabras y caracteres tiene \texttt{/home/<usuario>/tp/config}.

\end{enumerate}

\item Pipes

\begin{enumerate}

\item Liste en forma amplia los archivos del directorio \texttt{/usr/bin} que comiencen con la letra ``a''.
Del resultado obtenido, seleccione las l'ineas que contienen el texto \texttt{apt} e informe la cantidad de caracteres,
palabras y l'ineas.

Est'a prohibido, en este 'item, usar archivos temporales de trabajo.

\end{enumerate}

\end{enumerate}

\subsection{Scripting}

Escriba un script de shell que sincronice dos carpetas. El script debe recibir las dos carpetas (origen y destino, en ese
orden) desde la l'inea de comando, o preguntarlas interactivamente al usuario en caso de no recibirlas. Adem'as, debe
aceptar un modificador \texttt{-r} que indica modo de operaci'on recursivo.

Una vez conocidas las dos carpetas con las que se operar'a, el script deber'a copiar todos los archivos de la carpeta origen
que no est'en presentes en la carpeta destino, y adem'as tambi'en deber'a copiar todos los archivos presentes en ambas
carpetas que tengan una fecha de modificaci'on posterior en la carpeta origen que en la carpeta destino. De esta manera, al
ejecutar el script, estar'a seguro de que la carpeta destino contiene toda la informaci'on de la carpeta origen, excepto lo
que fue modificado posteriormente en la carpeta destino.

El modificador \texttt{-r} indica al script realizar la sincronizaci'on tambi'en con los archivos de todos los
subdirectorios de la carpetas origen y destino.

\paragraph{Sugerencia}

Recuerde que generalmente el esp'iritu de los scripts es proveer la m'inima l'ogica necesaria alrededor de otros programas
ya presentes en el sistema que puedan proveer parte de la funcionalidad requerida para resolver un determinado problema.

\subsection{Ejecuci\'on de procesos en background}

Antes de resolver esta secci'on instale los siguientes paquetes en la m'aquina virtual:

\begin{itemize}
\item \texttt{nano}: editor de texto.
\item \texttt{mc}: manejador de archivos.
\item \texttt{gcc}: compilador de C.
\item \texttt{libc6-dev}: biblioteca est'andar de C.
\end{itemize}

Cree el archivo \texttt{/home/<usuario>/tp/loop.c}. Comp'ilelo con \texttt{gcc}. El programa compilado debe llamarse
\texttt{loop}.

\CodigoF{loop.c}{loop.c}

\begin{enumerate}
\item Correrlo en foreground. ?`Qu'e sucede? Mate el proceso con \texttt{Ctrl-c}.
\item Ahora ejec'utelo en background: \texttt{/usr/src/loop > /dev/null \&}. Mate el proceso con el comando \texttt{kill}.
\end{enumerate}

\subsection{IPC y sincronizaci\'on}

\subsubsection{Pipes}

Muestre con un ejemplo en lenguaje C como realizar exclusi'on mutua entre dos procesos utilizando un s'olo pipe.

\paragraph{Sugerencia}

Revise la ayuda de la llamada al sistema \texttt{pipe} para construir el pipe y de \texttt{fork} para crear
nuevos procesos.

\subsubsection{Threads}

Antes de resolver este ejercicio instale el paquete \texttt{glibc-doc}.

Resuelva el problema de productor/consumidor utilizando threads.

\paragraph{Sugerencia}

Revise la ayuda de \texttt{pthreads}, la implementaci'on de threads en Linux, para conocer los mecanismos de creaci'on y
destrucci'on de threads. Adem'as, \texttt{pthreads} provee mecanismos de sincronizaci'on que le ayudar'an a resolver este
ejercicio.

\subsection{El kernel Linux}

Antes de resolver esta secci'on instale los siguientes paquetes en la m'aquina virtual:

\begin{itemize}
\item \texttt{make}: utilidad para mantener grupos de programas.
\item \texttt{linux-headers-<version>}: headers del kernel de Linux.
\end{itemize}

Sustituya \texttt{<version>} por el resultado del comando \texttt{uname -r}.

\subsubsection{Funcionamiento del kernel Linux}

\begin{enumerate}

\item Describa la administraci'on del procesador utilizada por defecto en el kernel Linux.

\item Describa la administraci'on de memoria utilizada por defecto en el kernel Linux.

\item Describa el sistema de archivos utilizado en la distribuci'on de Linux que instal'o en la m'aquina virtual.
?`Qu'e capas existen en el kernel Linux para soportar sistemas de archivos sobre dispositivos de bloques?

\end{enumerate}

\subsubsection{M'odulos de kernel}

El kernel de Linux permite ser ampliado en \textbf{runtime} con m'odulos. Los m'odulos son objetos de c'odigo compilado
que pueden ser insertados en runtime al kernel, siendo linkeados contra el kernel al momento de ser insertados. De esta
manera puede ampliarser la funcionalidad del kernel en runtime, sin tener que incluir todo el c'odigo en el binario
original.

\subsubsection{Compilando un m'odulo de kernel}

A continuaci'on compilaremos y probaremos un m'odulo de kernel muy simple. Este m'odulo simplemente escribe en la consola
``Hola kernel!'' al ser insertado y ``Chau, kernel.'' al ser removido. Cree el siguiente m'odulo en el archivo
\texttt{/home/<usuario>/tp/hello.c}.

\CodigoF{hello.c}{hello.c}

Para compilar este c'odigo deber'a construir una \textbf{Makefile}. Este archivo Makefile es utilizado luego por el
comando \texttt{make} para compilar el m'odulo con las opciones correctas. En el mismo directorio donde se encuentra
\texttt{hello.c} cree un archivo \texttt{Makefile} conteniendo:

\CodigoF{Makefile}{Makefile}

Luego ejecute:

\texttt{make}

Para compilar el m'odulo. Finalmente, pruebe insertar el m'odulo usando:

\texttt{insmod hello.ko}

Deber'ia ver el mensaje ``Hola kernel!'' en la consola. Quite el m'odulo usando:

\texttt{rmmod hello.ko}

Deber'ia ver el mensaje ``Chau, kernel''.

\subsubsection{Un m'odulo propio}

Escriba un m'odulo de kernel que permita controlar los LEDs del teclado \textbf{sin necesidad de escribir un programa 
en lenguaje C}. El m'odulo deber'a permitir prender o apagar cada LED usando simplemente comandos del shell.

Dado que la m'aquina virtual provista por VirtualBox no muestra el estado de los LEDs, la c'atedra provee abajo
un programa que lo hace. Compile utilizando:

\texttt{gcc -o check\_kbleds check\_kbleds.c}

\CodigoF{check\_kbleds.c}{check_kbleds.c}

\paragraph{Sugerencia}

Haga que su m'odulo cree un archivo en \texttt{/proc} y que las escrituras a ese archivo controlen los LEDs utilizando 
la IOCTL \texttt{KDSETLED} de la consola de Linux.
