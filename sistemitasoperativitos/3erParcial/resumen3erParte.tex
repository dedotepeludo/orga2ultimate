\documentclass[a4paper,spanish]{article}
\usepackage[spanish, activeacute]{babel}
\usepackage{fancyhdr}

\oddsidemargin 0in
\textwidth 6.2in
\topmargin 0in
\textheight 18cm
\addtolength{\topmargin}{-.5in}
%textheight 10in
\textheight 24cm
\parskip=1ex
\pagestyle{fancy}
\newcommand{\real}{\hbox{\bf R}}

\title{Resumen pa\'{}l tercer parcial}
\date{SiSop}
\author{Alfalfa producciones}


\begin{document}

\maketitle

\tableofcontents

\section{Comunicaci'on entre procesos}
La comunicaci'on entre procesos cuenta con 2 primitivas: SEND y RECIVE.
Estos pueden ser bloqueantes (sincr'onicos) o no bloqueantes (asincr'onicos).
Caracter'isticas a tener en cuentas son: la conexi'on y la confiabilidad. El
SEND puede tener un buffer.

Para ubicar un procedimiento remoto se puede tener un \emph{binding} din'amico
o uno est'atico.

\section{Procedimientos remotos}
\begin{itemize}
	\item asincr'onica: Cliente-Servidor
	\item sincr'onica: RPC
\end{itemize}

\subsection{RPC - Remote Procedure Call} 
\begin{enumerate}
	\item El procedimiento cliente le env'ia (v'ia \emph{stack}) en
pedido a su \emph{stub}
	\item el \emph{stub} cliente arma el mensaje y se lo env'ia al
\emph{kernel}
	\item \emph{kernel} cliente le env'ia el pedido al \emph{kernel} del
servidor
	\item el \emph{kernel} del servidor, se lo pasa a su \emph{stub}
	\item el \emph{stub} del servidor, se lo pasa al procedimiento del
servidor
	\item el procedimiento del servidor ejecuta y despu'es devuelve el
resultado por el camino inverso
\end{enumerate}

Las valores se pasan por \emph{copy/restore}. Cualquier otra forma tiene
problemas.

\subsubsection{Fallas en RPC}
\begin{itemize}
	\item No se encuentran (servidor y cliente) $\Rightarrow$
Excepci'on.
	\item Se pierde el requerimiento $\Rightarrow$ Retransmici'on
	\item Se pierde el mensaje de vuelta $\Rightarrow$ Retransmici'on
	\item Se cae el servidor antes de devolver la respuesta
$\Rightarrow$ Lo resuelve la pol'itica del servidor
	\item Se cae el servidor antes de procesar el pedido. Distintas
pol'iticas:
		\begin{itemize}
			\item \emph{at least once} operaciones idempotentes
			\item \emph{at most once} 
			\item \emph{exactly once} muy dificil de lograr
		\end{itemize}
	\item Se cae el cliente produciendo huerfanos $\Rightarrow$
		\begin{description}
			\item[Reencarnaci'on] Al levantarse hace
\emph{broadcast} avisando que volvi'o y quien era y cada servidor mata los
procesos huerfanos
			\item[Reencarnaci'on suave] Cuando llega un
\emph{broadcast} avisando que volvi'o una m'aquina, cada servidor verifica si
sus computos remotos siguen teniendo due'no.
			\item[Expiraci'on Exterminaci'on] el \emph{stub}
cliente lleva un \emph{log} de los pedidos que se hicieron. Al levantarse mata
explicitamente dichos procesos.
		\end{description}
\end{itemize}

\subsection{Sockets}
Hay un \emph{socket} con nombre en el servidor y uno sin nombre en el cliente.

El servidor crea un \emph{socket} con nombre y espera una conexi'on.

``Un \emph{socket} es un descriptor con varias cositas adentro''

Hay con conexi'on y sin conexi'on.
\subsubsection{Socket con conexi'on}
\begin{itemize}
	\item El servidor abre el canal, publica su direcci'on y se pone a
esuchar en busca de nuevas conexiones
	\item El cliente crea un socket y pide conectarse con el servidor
	\item El servidor acepta la conexi'on, entonces se lleva a cabo
	\item El cliente escribe el pedido
	\item El servidor lee el pedido, lo resuelve y escribe los dato al
cliente
	\item El cliente lee los datos y cierra el canal
\end{itemize}
La comunicaci'on ser realiza con WRITE y READ

\subsubsection{Socket sin conexi'on}
\begin{itemize}
	\item El servidor abre el canal, publica su direcci'on y se bloquea
esperando que llegue informaci'on
	\item El cliente crea un socket y pide conexi'on (ubicar su direcci'on
en el \emph{binding})  con el servidor y env'ia el pedido
	\item El servidor resuelve y env'ia los dato al cliente
	\item El cliente recibe los datos y cierra el canal
\end{itemize}
La comunicaci'on ser realiza con SEND y RECEIVE


\section{Dise'no de un sistema operativo}
\subsection{Transaparencia}
\begin{description}
	\item[Locaci'on] los usuarios no deben saber donde est'an los recursos
	\item[Migraci'on] los recursos pueden cambiar de migraci'on sin cambiar su nombre
	\item[Replica] los usuarios no pueden decir la cantidad de copias existentes. i.e. varias copias de un archivo
	\item[Concurrencia] varios usuarion pueden compartir recursos de manera autom'atica. El SO asegura acceso secuencial no concurrente
	\item[Paralelismo] Las actividades pueden ocurrir paralelamente sin que los usuarios lo sepan.
\end{description}

\subsection{Flexibilidad}
\subsubsection{Kernel monol'itico}
Tiene la mayor'ia de los servicio. Centralizado, con capacidades red e integraci'on de servicios remotos. Las llamadas al sistema se realizan mediante interrupciones
\subsubsection{Microkernel}
Provee lo m'inimo indispensable, el grueso de los servicios los carga el sistema operativo. Proporciona 4 servicios:
\begin{itemize}
	\item Un mecanismo de comunicaci'on entre procesos
	\item Cierta administraci'on de memoria
	\item Una cantidad limitada de planificaci'on y administraci'on de procesos de bajo nivel
	\item Entrada/Salidad de bajo nivel
\end{itemize}

\subsection{Confiabilidad}
Debe ser altamaente disponible, y debe cuidar bien los datos que se le dan.
Adem'as de ser seguro y tolerante a fallas.

\subsection{Performance}
\begin{itemize}
	\item Tiempo de respuesta
	\item Rendimiento (medido en cantidad de trabajos por hora)
	\item Uso del sistema
	\item Cantidad consumida de la capacidad de red
\end{itemize}

\subsection{Escalabilidad}
NFS no es escalable.

\section{TCP/IP - OSI/ISO}
\begin{tabular}{|c|c|c|c|}
\hline
&TCP/IP & & OSI/ISO\\ \hline
 & 		& 7 & Aplicaci'on\\
5& Aplicaci'on 	& 6 & Presentaci'on\\
 & 	 	& 5 & Sesi'on\\
4& Transporte 	& 4 & Transporte\\
3& Internet	& 3 & Red\\
2& Interfaz de red &2& Enlace de datos\\
1& Hardware	& 1 & F'isica\\ \hline
\end{tabular}

Capas:
\begin{description}
	\item[F'isica] maneja bits
	\item[Enlace] maneja paquetes, detecta errores de transmici'on por checksum.
	\item[Red] se encarga del \emph{ruteo}  y asegura la trayectoria. La anterior s'olo dirige a LAN, esta sirve para direcciones ```m'as grandes''.
	\item[Transporte] mantiene el orden de los paquetes para que tenga sentido el mensaje
	\item[Sesi'on] es un refinado de la de transporte, sabe con qui'en es la comunicaci'on y ayuda a la sincronizaci'on
	\item[Presentaci'on] se encarga de homogeneizar los datos, facilitando la comunicaci'on entre m'aquinas con representaciones distintas.
	\item[Aplicaci'on] interact'ua con los usuario
\end{description}

\section{Rendez-Vous}
Asim'etrica-Semi sincr'onica

Semisincr'onica: send no bloqueantes y receive bloqueante

\section{Mailbox}
Desacopla el emisor y receptor

Mayor flexibilidad en mensajes.

Desventaja: centraliza.

Modalidades:
\begin{itemize}
	\item uno a uno
	\item muchos a uno (cliente servidor)
	\item uno a muchos (broadcast)
	\item muchos a muchos (comunicaci'on entre servers)
\end{itemize}
Colas con prioridad.



\section{Sincronizaci'on}

\subsection{Relojes}
Se usan para la entrega de mensajes ``a lo sumo una vez'' y para mantener
consistencia en las caches de los clientes.

L'ogicos, cada proceso tiene uno propio. Lamport, los tiempos concuerdan con el
orden de los procesos.

F'isicos, hay varias versiones para sincronizar relojes de un sistema
distribuido:
\begin{description}
	\item[Cristian] tiene un reloj universal en un servidor de tiempo
(WWV), cada m'aquina se encarga de preguntar y se acomoda a ese tiempo
	\item[Berkeley] cada m'aquina env'ia su tiempo a un servidor y este
responde con la correcci'on que hay que realizar promediando todas las horas
recibidas no tiene ning'un WWV
	\item[Promedios] todos hacen \emph{broadcast} de su hora y a su vez,
todos leen las horas de los dem'as, las promedian y se quedan con ese dato
	\item[Multiples fuentes externas] parecido al anterior, salvo que se
queda con la intersecci'on entre todas las se'nales de los tiempos recibidos
\end{description}

\subsection{Exclusi'on mutua}
Hay varias versiones:
\begin{description}
	\item[centralizado] Existe un coordinador, su caida implica que nadie
m'as entre a zonas de uso exlusivo. Se requieren 3 mensajes para el uso de una
zona dedicada (pedido, ack, liberado).
	\item[distribuido] (Ricart-Agrawala) requiere sincronizaci'on de los
relojes l'ogicos. Un nodo pregunta a todos, y hasta que no obtiene todos los OK
no entra en la zona cr'itica, si un nodo est'a usando no responde. Al liberar
se avisa a todos. Empates se definen por menor marca temporal. Hay problema si
se cae un nodo porque no responde. Se puede mejorar haciendo que todos
contesten o que se acceda por mayor'ia simple. Tiene $n$ puntos de falla. Todos
los nodos tienen que tener una lista de procesos activos. Se requieren
$2(n-1)$ mensajes para acceder en la primer versi'on.
	\item[token-ring] red en forma de anillo, se pasa un token, el que
tiene el token tiene permiso. Un problema es la perdida del token, porque se
cae el que lo ten'ia o porque es lento. Requiere entre 1 y n mensajes
\end{description}

\subsection{Recuperaci'on de caidas del coordinador}
\begin{description}
	\item[Bully] El que detecta que se cay'o el coordinador manda un
mensaje a cada nodo con n'umero mayor a 'el. Si alguno responde, entonces, ese
repite la operatoria hasta que alguno no reciba ninguna respuesta, se considera
que ese es el de mayor n'umero y asume como nuevo coordinador avisando a todos.
Cuando se restaura el coordinador caido reasume su puesto.
	\item[Anillo] Parecido al anterior, s'olo que se pasa una lista por un
anillo y se van agregando los nodos vivos. Al dar toda la vuelta el que inici'o
este mecanismo, decide cu'al es el mayor y ese es el que asume como nuevo
coordinador.
\end{description}

\section{Transacciones}
Todo o nada. 

Primitivas: at'omicas (indivisible), consistente, serializable (las concurrentes no
interfieren), permanentes (una vez realizado el commit los cambios son
permanentes).

\subsection{Implementaci'on de las transacciones}
\begin{itemize}
	\item En espacio privado de trabajo, se copia todo, en caso de rollback
se borra el espacio de trabajo, en caso de commit se reapuntan los punteros de
los datos.
	\item Log de grabaci'on adelantada. Graba el valor viejo y el nuevo en el
log y luego graba el dato en memoria. El commit s'olo escribe ``commit'' en el
log. En el rollback deshace los cambios (es el cuello de botella). Es m'as
tolerante a caidas.
\end{itemize}

\subsubsection{Subtransacciones}
La transacci'on madre dispara transacciones hijas. El commit de las
subtransacciones s'olo es permanente en el universo de la transacci'on madre.
En caso de que la transacci'on madre aborte o haga rollback los datos afectados
por las subtransacciones no se ven fuera.

\subsubsection{Two-phase commit}
Cuando una transacci'on involucra varios procesos nodos, se tienen que poner
todos ``de acuerdo'' para realizar el commit. Hay un coordinador, que en
general suele ser el que inicia la transacci'on.

Todo se escribe en el log y se env'ian como \emph{mensaje}
\begin{enumerate}
	\item (INICIO FASE 1) El coordinador escribe ``preparado'' a todos los
subordinados
	\item Todos los subordinados, cuando lo est'an, env'ian ``listo''
	\item El coordinador revisa que todas las respuestas sean ``listo''
(FIN FASE 1)
	\item (INICIO FASE 2) El coordinador escribe ``commit''
	\item Los subordinados escriben ``commit'' y lo realizan
	\item Los subordinados env'ian ``terminado'' (FIN FASE 2)
\end{enumerate}

Posibles problemas:
\begin{itemize}
	\item Si se cae el coordinador luego de enviar ``commit'', el volver
ve las respuestas en el log.
	\item Si se cae el subordinado ve la decisi'on en el log
	\item Si un subordinado no responde, se produce un \emph{timeout} y se
decide si cancela o no.
	\item si se cae el coordinador luego de escribir el log, al volver
puede seguir.
\end{itemize}

\subsection{Control de concurrencia}
\subsubsection{Bloqueo}
Depende de la granularidad. Lectura (varios concurrentes). Escritura (s'olo
uno). Es costoso y no est'a libre de deadlock.

\subsubsection*{Bloqueo de dos etapas}
Reduce la posibilidad de deadlock. Hay una etapa de pedir recursos hasta que
llega al punto de bloqueo (tiene todos los recursos). Puede usarse
\emph{hold\&wait}, timeout y otros para resolver posibles bloqueos.

Si las transacciones usan esto son serializables.

\subsubsection{Control optimista}
Otorga todos los permisos, al momento de hacer el commit verifica si todo est'a
como al iniciar la transacci'on, si algo cambi'o no se hace commit. Est'a libre
de \emph{deathlock}. Permite gran paralelismo. Anda muy bien con espacio
privado de trabajo.

\subsubsection{Sellos temporales}
Cada transacci'on tiene un sello temporal para lectura y escritura. Cuando una
transacci'on encuentra un sello mayor qe el suyo, aborta. Est'a libre de
\emph{deathlock} pero es dificil de implementar.

\subsection{Deathlock}
La detecci'on en un sistema distribuido puede dar falsos anuncios por la demora
de los mensajes.

Se puede hacer detecci'on enviando un mensaje que si llega a volver al que lo
envi'o este se suicida. El mensaje tiene un id de proceso original y los
recursos que est'an involucrados.

\subsubsection{Formas de prevenci'on}
\begin{description}
	\item[1 recurso por vez]
	\item[Asignaci'on total]
	\item[Pedido ordenado de recursos]
	\item[Wound-Wait (golpea y espera)] El viejo desaloja al nuevo. El
joven espera al viejo. Permite inanici'on.
	\item[Wait-Die (espera y muere)] El viejo espera un recurso que tiene
un joven. El joven que quiere un recurso que tiene un viejo muere y debe
empezar de nuevo (sin cambiar su sello temporal). Permite inanici'on.
\end{description}

\section{Caches en los clientes}
\subsection{Sin reloj}
Se negocian todos los permisos. Si se decide invalidar algo se avisa. Bloquea.

\subsection{Con reloj}
Al acceder se le da fecha de expiraci'on. Si otro lo pide espera expiraci'on o
cancela permiso.
	
\section{Filesystem}
Modelos:
\begin{itemize}
	\item Upload/Download
	\item Remote Access
	\item Transparencia de nombres
\end{itemize}

\subsection{Sem'antica de archivos compartidos}
\begin{description}
	\item[Unix] cambios instantaneos (cada 3 segundos se refresca)
	\item[Sesi'on] se guardan los cambios al finalizar
	\item[Archivos inmutables] s'olo se permite crea y leer (tipo cintas)
	\item[Transacciones] son de tipo sesi'on.
\end{description}

\subsection{Tipos de servidor}
\begin{tabular}{|c|c|}
\hline
Servidor con estado		& Servidor sin estado \\ \hline
Permite lectura adelantada	& No necesita open/close \\ \hline
Hay lock de archivos		& Sin limite de cantidad de archivos abiertos
\\ \hline
Mejor desempe'no		& M'as tolerante a fallas \\ \hline
F'acil idempotencia		& No desperdicia espacio en tablas \\ \hline
Solicitudes m'as cortas		& Solicitudes m'as largas \\ \hline
Cuando se cae se pierde todo. & Si falla el cliente no se pierde nada. \\
La recuperaci'on queda a cargo de los clientes & No importa \\ \hline
Si el cliente se cae el servidor no & Si se cae
el cliente todo ok. \\
sabe como eliminar las entradas & \\
\hline
\end{tabular}

\subsection{Caching}
Distintos niveles. Problemas de consistencia.

Metodos:
\begin{description}
	\item[Escritura a traves de cache] funciona. No afecta el tr'afico de
escritura
	\item[Escritura demorada] Mejor desempe'no. Posible sem'antica ambigua.
	\item[Escritura al cierre] Sem'antica de sesi'on.
	\item[Control centralizado] Sem'antica Unix. No robusto. Poco
escalable.
\end{description}

\subsection{Implementaciones}
\begin{itemize}
	\item RFS - Remote File System
	\item NFS - Network File System
	\item AFS - Andrew File System
\end{itemize}

\subsubsection{RFS}
\begin{itemize}
	\item Estado en el cliente y en el servidor
	\item Todo RPC
	\item Caida del servidor afecta al cliente
	\item Permite lecturas adelantadas
	\item Control de acceso centralizado
	\item Existe una cantidad m'axima de archivos en operaci'on (abiertos
al mismo tiempo)
\end{itemize}

\subsubsection{NFS (mount protocol)}
\begin{itemize}
	\item El servidor, si tiene el acceso permitido, devuelve un file handle.
	\item El file handle identifica el archivo que guarda. En Unix es un id
y un nro. de i-nodo para identificar el directorio montaod dentro del FS
exportado
	\item El servidor tiene una lista de los cliente y las cosas que
montaron
\end{itemize}

\begin{tabular}{|c|c|}
\hline
Servidor			& Cliente \\ \hline
Exporta FS		& Monta archivos y directorios \\
R/W seg'un diga el cliente	& R/W que pide al NFS que realice \\
Sin estado			& Con estado \\
Comunicaci'on por RPC		& Comunicaci'on por RPC \\
Sin Cach'e			& Con Cache \\
Exporta FS y registra lo exportado y da un puntero al cliente & Con el handler
tiene un inodo-v\\
\hline
\end{tabular}

\subsubsection{AFS}
\begin{itemize}
	\item Ocultamiento (sesi'on)
	\item Espacio 'unico de nombres
	\item El cliente se conecta por red Venus al servidor (red de
servidores llamada Vice) pidiendo un archivo
	\item Cada cliente tiene un servidor, pero este 'ultimo est'a conectado a
otros servidores
	\item Entre Venus y Vice hay un identificador llamado fid que tiene el
n'umero de volumen, el n'umero de vnodo y un c'odigo 'unico
\end{itemize}

\end{document}
