Aqu'i haremos uso del comando \textit{renice}, que sirve para
cambiar la prioridad de un proceso que est'e en ejecuci'on. Sus
par'ametros son:

\begin{itemize}

\item el valor ``nice'', que representa la prioridad que queremos
asignar, y es un valor entero entre -20 y 19 (-20 es la mayor
prioridad y 19, la menor)

\item el PID del proceso al que le queremos modificar la prioridad

\end{itemize}

En la siguiente captura, mostramos como ejecutamos tres procesos
iguales, y hacemos que uno tome mayor prioridad con el comando que
nombramos en este apartado. Esto lo notamos a partir del comando
$ps$ $-l$, que nos muestra como el proceso ``loop3'', a los 6
minutos de ejecuc�'on, tom'o 5 minutos del procesador (columna TIME)
y est'a ocupando un 70\% del procesador (columna C).

\begin{envCodigo}
gr01@mondiola:~/tp$ gcc -o loop1 loop.c
gr01@mondiola:~/tp$ gcc -o loop2 loop.c
gr01@mondiola:~/tp$ gcc -o loop3 loop.c
gr01@mondiola:~/tp$ ps -l
F S   UID   PID  PPID  C PRI  NI ADDR SZ WCHAN  TTY          TIME CMD
0 R  1000  5143  5141  0  80   0 -  1764 -      pts/0    00:00:01 bash
0 R  1000  5208  5143  0  80   0 -   607 -      pts/0    00:00:00 ps
gr01@mondiola:~/tp$ ./loop1 > /dev/null &
[1] 5209
gr01@mondiola:~/tp$ ./loop2 > /dev/null &
[2] 5210
gr01@mondiola:~/tp$ ./loop3 > /dev/null &
[3] 5211
gr01@mondiola:~/tp$ ps -l
F S   UID   PID  PPID  C PRI  NI ADDR SZ WCHAN  TTY          TIME CMD
0 R  1000  5143  5141  0  80   0 -  1764 -      pts/0    00:00:01 bash
0 R  1000  5209  5143 49  80   0 -   391 -      pts/0    00:00:08 loop1
0 R  1000  5210  5143 33  80   0 -   391 -      pts/0    00:00:04 loop2
0 R  1000  5211  5143 31  80   0 -   391 -      pts/0    00:00:02 loop3
0 R  1000  5212  5143  0  80   0 -   607 -      pts/0    00:00:00 ps
gr01@mondiola:~/tp$ sudo renice -15 5211
[sudo] password for gr01:
5211: old priority 0, new priority -15
gr01@mondiola:~/tp$ ps -l
F S   UID   PID  PPID  C PRI  NI ADDR SZ WCHAN  TTY          TIME CMD
0 R  1000  5143  5141  0  80   0 -  1766 -      pts/0    00:00:01 bash
0 R  1000  5209  5143 22  80   0 -   391 -      pts/0    00:00:54 loop1
0 R  1000  5210  5143 21  80   0 -   391 -      pts/0    00:00:49 loop2
0 R  1000  5211  5143 51  65 -15 -   391 -      pts/0    00:02:00 loop3
0 R  1000  5243  5143  0  80   0 -   607 -      pts/0    00:00:00 ps
gr01@mondiola:~/tp$ ps -l
F S   UID   PID  PPID  C PRI  NI ADDR SZ WCHAN  TTY          TIME CMD
0 S  1000  5143  5141  0  80   0 -  1766 -      pts/0    00:00:01 bash
0 R  1000  5209  5143 13  80   0 -   391 -      pts/0    00:01:01 loop1
0 R  1000  5210  5143 12  80   0 -   391 -      pts/0    00:00:56 loop2
0 R  1000  5211  5143 70  65 -15 -   391 -      pts/0    00:05:08 loop3
0 R  1000  5244  5143  0  80   0 -   607 -      pts/0    00:00:00 ps

gr01@mondiola:~/tp$ killall loop*
loop: no process killed
loop.c: no process killed
[1]   Terminated              ./loop1 > /dev/null
[2]-  Terminated              ./loop2 > /dev/null
[3]+  Terminated              ./loop3 > /dev/null
\end{envCodigo}
