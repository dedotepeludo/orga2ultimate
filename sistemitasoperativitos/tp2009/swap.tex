
Para aumentar el tama\~{n}o de la partici'on swap, crearemos un
nuevo archivo de este tipo y lo anexaremos al sistema.

Para esto, utilizaremos las funciones:

\begin{itemize}
\item dd para crear una copia de la swap existente en otro archivo
\item mkswap para definir el archivo nuevo como de tipo swap
\item swapon para anexar el archivo creado al sistema
\end{itemize}

\begin{envCodigo}
gr01@mondiola:~$ free
             total       used       free     shared    buffers     cached
Mem:        513624     234316     279308          0       8800     107720
-/+ buffers/cache:     117796     395828
Swap:       200772          0     200772
gr01@mondiola:~$ sudo dd if=/dev/zero of=/mnt/coco bs=1024 count=1024
[sudo] password for gr01:
1024+0 records in
1024+0 records out
1048576 bytes (1.0 MB) copied, 0.0105516 s, 99.4 MB/s
gr01@mondiola:~$ sudo mkswap /mnt/coco
Setting up swapspace version 1, size = 1044 kB
no label, UUID=a022aa86-6ac3-450f-8e9a-107066e46c46
gr01@mondiola:~$ sudo swapon /mnt/coco
gr01@mondiola:~$ free
             total       used       free     shared    buffers     cached
Mem:        513624     235528     278096          0       8920     108888
-/+ buffers/cache:     117720     395904
Swap:       201788          0     201788

gr01@mondiola:~$ cat /proc/swaps
Filename                Type        Size    Used    Priority
/dev/sda5                               partition   200772  0   -1
/mnt/coco                               file        1016    0   -2
\end{envCodigo}
