Los \textit{hardlinks} crean enlaces al \textit{inodo} del archivo
al que queremos apuntar, y estos 'ultimos apuntan al espacio de
datos donde reside el archivo.

Seg'un el est'andar POSIX\footnote{Portable Operating System
Interface. La X viene de UNIX como se�a de identidad de la API.}, se
establece un modelo de sistema de archivos que se ajusta al empleado
en los UNIX tradicionales. La estructura contiene los siguientes
datos:
\begin{itemize}

\item La longitud del archivo en bytes

\item El identificador de usuario, del creador o un propietario del archivo con derechos diferenciados

\item El identificador de grupo de un grupo de usuarios con derechos diferenciados

\item El modo de acceso: capacidad de leer, escribir, y ejecutar el archivo por parte del propietario, del grupo y de otros usuarios

\item Las marcas de tiempo con las fechas de 'ultima modificaci'on (mtime), acceso (atime) y de alteraci'on del propio inodo (ctime).

\item El n'umero de enlaces, esto es, la cantidad de nombres (entradas de directorio) asociados con este inodo.

\end{itemize}


Con respecto a porqu'e se pueden borrar los \textit{hardlinks} sin
borrar el archivo, podemos aducir que es porque el proceso de
eliminaci'on de este tipo de enlaces, solo desvincula un nombre de
los datos reales. Los datos todav'i estar'an accesibles mientras
quede alg'un enlace y reci'en cuando se eliminase el 'ultimo enlace
duro, el espacio que ocupaban los datos se considerar'a disponible.
