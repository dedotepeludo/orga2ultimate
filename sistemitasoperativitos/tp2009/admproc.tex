PC o Process Scheduler (Programador de Procesos) es en realidad el
coraz'on del Sistema Operativo.

Sus responsabilidades son:

\begin{itemize}

\item Permitirle a los procesos crear nuevas copias de s'i mismos (forking)

\item Determinar qu'e proceso tendr'a acceso a la CPU

\item Recibir interrupciones y desviarlas hacia el subsistema respectivo

\item Enviar mensajes a los procesos de usuario

\item Manejar el reloj de Hardware (Hardware Timer)

\item Liberar recursos cuando un programa los haya desocupado

\end{itemize}

El Process Scheduler tambi'en soporta modulos cargados
din'amicamente. Los m'as importantes son el ``Virtual File System''
y ``Network Interface'' (explicados en puntos posteriores).

Tambi'en provee dos interfaces. Una, provee una interface de llamada
a sistema limitada que los procesos pueden llamar. Dos, provee una
interface para el resto del espacio del Kernel.

En tiempo de booteo, Linux posee s'olo un proceso, $init()$, que a
su vez realiza copias de s'i mismo a trav'es de llamadas $fork()$.
Cuando un proceso termina, ejecuta la llamada $a\_exit()$.

PS ademas posee llamados a se�ales a trav'es de $signal()$. Esta
llamada permite a un proceso asociar un manejador de funcion
(function handler) con una se�al en particular. Las se�ales pueden
considerarse como una forma de IPC o intercomunicaci'on de procesos.

Acerca del timer, Linux maneja un reloj interno que se inicia cada
10 milisegundos. Esto permite que los procesos se reasignen
(Scheduling) cada 10 ms. Esta unidad se llama ``clock tick'' y sirve
para medir la cantidad de ticks que un proceso en particular puede
continuar ejecut�ndose sin forzar su reasignaci'on.

Cada proceso se asigna con un identificador 'unico, llamado Process
ID o pid, que es asignado a la lista de procesos o tabla de
procesos.

Finalmente, una estructura modular es usada para representar los
modulos cargados en tiempo de ejecuci'on. Esta estructura posee
campos que son usados para implementar una lista de estructura de
modulos. Es decir, un campo que apunta a la tabla de s�mbolos de los
m'odulos y otro con su nombre.
