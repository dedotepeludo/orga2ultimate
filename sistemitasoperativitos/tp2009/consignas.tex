\section{Consignas}

Antes de empezar, ejecute:

\texttt{sudo apt-get install man-db manpages manpages-dev}


De esta manera tendr'a acceso a ayuda en l'inea ejecutando:

\texttt{man <comando>}

Por ejemplo:

\texttt{man cp}

Puede adem'as instalar la versi'on en castellano de la ayuda ejecutando:

\texttt{sudo apt-get install manpages-es}

Para acceder a la ayuda en castellano ejecute por ejemplo:

\texttt{man -L es cp}

Tenga presente que no todos los comandos poseen ayuda en castellano.

\texttt{sudo} permite a usuarios normales ejecutar comandos que requieren permisos de administrador.
Al ejecutar un comando con \texttt{sudo} el sistema le pedir'a su password, y no el password del administrador
(llamado \texttt{root} en Linux, siguiendo la tradici'on de Unix). Esto sucede ya que el sistema permite que
ciertos usuarios (que deber'ian corresponderse con usuarios ``privilegiados'' del sistema) puedan utilizar \texttt{sudo}
ingresando solamente su propio password. El usuario por defecto creado en una instalaci'on de Ubuntu tiene este permiso
y por lo tanto en Ubuntu no es necesario una cuenta de administrador o \texttt{root}.

\texttt{apt-get} es el manejador de paquetes de la distribuci'on Ubuntu. Permite instalar, actualizar y desinstalar
programas. M'as adelante lo utilizaremos para instalar las herramientas necesarias para compilar programas en
Linux.

Si se encontrara detr'as de un proxy, antes de utilizar \texttt{apt-get} debe configurar el proxy. Ejecute el siguiente
comando:

\small
\texttt{sudo echo ''Acquire::http::Proxy $\backslash$''http://proxy.uba.ar:8080$\backslash$'';'' > /etc/apt/apt.conf}
\normalsize

\subsection{Comandos b\'asicos de Unix}

\begin{enumerate}

\item \texttt{pwd} Indique qu'e directorio pasa a ser su directorio actual si ejecuta:

\begin{envRespuesta}
gr01@mondiola:~\$ pwd\\
/home/gr01
\end{envRespuesta}

\begin{enumerate}
\item \texttt{cd /usr/bin}

\begin{envRespuesta}
gr01@mondiola:~\$ cd /usr/bin/\\
gr01@mondiola:/usr/bin\$ pwd\\
/usr/bin
\end{envRespuesta}

\item \texttt{cd}

\begin{envRespuesta}
gr01@mondiola:/usr/bin\$ cd\\
gr01@mondiola:~\$ pwd\\
/home/gr01
\end{envRespuesta}

\item ?`C'omo explica el punto anterior?

\begin{envRespuesta}Al utilizar el comando cd sin par'ametros, si existe la variable
HOME entonces se hace un chdir a ese path.
\end{envRespuesta}

\end{enumerate}

\item \texttt{cat} ?`Cu'al es el contenido del archivo \texttt{/home/<usuario>/.profile}?

\begin{envCodigo}
gr01@mondiola:~$ cat .profile 
# ~/.profile: executed by the command interpreter for login shells.
# This file is not read by bash(1), if ~/.bash_profile or ~/.bash_login
# exists.
# see /usr/share/doc/bash/examples/startup-files for examples.
# the files are located in the bash-doc package.

# the default umask is set in /etc/profile
#umask 022

# if running bash
if [ -n "$BASH_VERSION" ]; then
    # include .bashrc if it exists
    if [ -f "$HOME/.bashrc" ]; then
        . "$HOME/.bashrc"
    fi
fi

# set PATH so it includes user's private bin if it exists
if [ -d "$HOME/bin" ] ; then
    PATH="$HOME/bin:$PATH"
fi
\end{envCodigo}

\item \texttt{find} Liste \textbf{todos} los archivos que comienzan con \texttt{vmlinuz}.

\begin{envRespuesta}
gr01@mondiola:~\$ sudo find / -name *vmlinuz* \\
$[$sudo$]$ password for gr01: \\
/boot/vmlinuz-2.6.24-23-virtual\\
/boot/vmlinuz-2.6.24-24-virtual\\
/vmlinuz.old
\end{envRespuesta}

Estos archivos son im'agenes del kernel Linux.

\item \texttt{mkdir} Genere un directorio \texttt{/home/<usuario>/tp}.

\begin{envRespuesta}
gr01@mondiola:~\$ mkdir tp
\end{envRespuesta}

\item \texttt{cp} Copie el archivo \texttt{/etc/passwd} al directorio \texttt{/home/<usuario>/tp}.

\begin{envRespuesta}
gr01@mondiola:~\$ cp /etc/passwd tp/
\end{envRespuesta}

\item \texttt{chgrp} Cambie el grupo del archivo \texttt{/home/<usuario>/tp/passwd} para que sea el suyo.

\begin{envRespuesta}
El archivo ya pertenece a gr01, igualmente se puede hacer:\\
gr01@mondiola:~\$ chgrp gr01 tp/passwd
\end{envRespuesta}

\item \texttt{chown} Cambie el due'no del archivo \texttt{/home/<usuario>/tp/passwd} para que sea su usuario.

\begin{envRespuesta}
El archivo ya es de gr01, pero en caso de tener que hacerlo: 
gr01@mondiola:~\$ chown gr01 tp/passwd
\end{envRespuesta}

\item \texttt{chmod} Cambie los permisos del archivo \texttt{/home/<usuario>/tp/passwd} para que:

\begin{envRespuesta}
Antes de empezar el pr'oximo paso: \\
gr01@mondiola:~\$ chmod 000 tp/passwd \\
para limpiar los permisos originales.
\end{envRespuesta}
\begin{itemize}
\item el propietario tenga permisos de lectura, escritura y ejecuci'on

\begin{envRespuesta}
gr01@mondiola:~\$ chmod u+r+w+x tp/passwd 
\end{envRespuesta}

\item el grupo tenga s'olo permisos de lectura y ejecuci'on

\begin{envRespuesta}
gr01@mondiola:~\$ chmod g+r+x tp/passwd 
\end{envRespuesta}

\item el resto tenga s'olo permisos de ejecuci'on

\begin{envRespuesta}
gr01@mondiola:~\$ chmod o+x tp/passwd 
\end{envRespuesta}
\end{itemize}

\item \texttt{grep}

Muestre las l'ineas que tienen el texto ``localhost'' en el archivo \texttt{/etc/hosts}.

\begin{envCodigo}
gr01@mondiola:~\$ grep localhost /etc/hosts
127.0.0.1	localhost
::1     ip6-localhost ip6-loopback
\end{envCodigo}

Muestre todas las l'ineas que tengan el texto ``POSIX'' de todos los archivos (incluyendo subdirectorios) en \texttt{/etc}.
Evite los archivos binarios y aquellos archivos y directorios que no tienen permiso de lectura para su usuario.

\begin{envCodigo}
gr01@mondiola:~\$ grep -R -I -s POSIX /etc/
/etc/init.d/glibc.sh:                echo WARNING: POSIX threads library NPTL 
                                     requires kernel version
/etc/security/limits.conf:#        - msgqueue - max memory used by POSIX 
                                     message queues (bytes)
\end{envCodigo}

\item \texttt{passwd} Cambie su password.

\begin{envCodigo}
gr01@mondiola:~$ passwd 
Changing password for gr01.
(current) UNIX password:
Enter new UNIX password:
Retype new UNIX password:
passwd: password updated successfully
\end{envCodigo}

\item \texttt{rm} Borre el archivo \texttt{/home/<usuario>/tp/passwd}

\begin{envRespuesta}
gr01@mondiola:~\$ rm tp/passwd 
\end{envRespuesta}
\item \texttt{ln}

Enlazar el archivo \texttt{/etc/passwd} a los archivos \texttt{/tmp/contra1} y \texttt{/tmp/contra2}.

Hacer un \texttt{ls -l} para ver cuantos enlaces tiene \texttt{/etc/passwd}.

\begin{envCodigo}
gr01@mondiola:~$ ln /etc/passwd /tmp/contra1
gr01@mondiola:~$ ln /etc/passwd /tmp/contra2
gr01@mondiola:~$ ls -l /etc/passwd
-rw-r--r-- 3 root root 1544 2009-06-14 15:03 /etc/passwd
\end{envCodigo}

Estos enlaces se llaman ``hardlinks''. Cada nuevo enlace referencia el mismo espacio ocupado del disco r'igido,
y por lo tanto cada hardlink es igual de representativo de esos bytes ocupados del disco r'igido.
El espacio ocupado solamente se liberar'a cuando todos los enlaces hayan sido borrados.

Ahora enlace el archivo \texttt{/etc/passwd} de manera ``soft'' al archivo \texttt{contra3}.

Verifique con \texttt{ls -l} que no aument'o la cantidad de enlaces de \texttt{/etc/passwd}.

\begin{envCodigo}
gr01@mondiola:~$ ln -s /etc/passwd /tmp/contra3
gr01@mondiola:~$ ls -l /etc/passwd
-rw-r--r-- 3 root root 1544 2009-06-14 15:03 /etc/passwd
gr01@mondiola:~$ ls -l /tmp/contra*
-rw-r--r-- 3 root root 1544 2009-06-14 15:03 /tmp/contra1
-rw-r--r-- 3 root root 1544 2009-06-14 15:03 /tmp/contra2
lrwxrwxrwx 1 gr01 gr01   11 2009-06-16 00:48 /tmp/contra3 -> /etc/passwd
\end{envCodigo}

Estos enlaces se llaman ``softlinks'' y apuntan no a los bytes del disco r'igido sino a la ruta del archivo a ser enlazado.
Operar sobre el softlink es igual que operar sobre el archivo, sin embargo los softlinks no cuentan en la cantidad de
enlaces (ya que no apuntan a los bytes ocupados del disco r'igido) y pueden ser borrados sin afectar al archivo original,
aunque si se borra el archivo original el softlink quedar'a hu'erfano y no apuntar'a a nada.

\item \texttt{mount}

Monte el CD-ROM de instalaci'on de Ubuntu JeOS y liste su contenido.

Para hacer esto deber'a especificar la ISO de instalaci'on de Ubuntu JeOS como CD-ROM de la m'aquina virtual.
Si bien puede hacer esto como lo hizo para instalar el sistema, si la m'aquina virtual est'a corriendo debe hacer click
derecho en el 'icono con forma de CD-ROM en la esquina inferior derecha de la m'aquina virtual, y seleccionar
\textbf{CD/DVD-ROM Image...} (ver figura \ref{isocdrom3}). En la ventana que aparece seleccione la ISO de instalaci'on
(ver figura \ref{isocdrom4}).

%\Imagen{vb_ubuntu/vb_isocdrom3.png}{7cm}{Usando una imagen ISO con la m'aquina virtual corriendo.}{isocdrom3}
%\Imagen{vb_ubuntu/vb_isocdrom4.png}{7cm}{Seleccionando la ISO de instalaci'on.}{isocdrom4}

Presente los filesystems que tiene montados.

\begin{envCodigo}
gr01@mondiola:~$ mount
/dev/sda1 on / type ext3 (rw,relatime,errors=remount-ro)
proc on /proc type proc (rw,noexec,nosuid,nodev)
/sys on /sys type sysfs (rw,noexec,nosuid,nodev)
varrun on /var/run type tmpfs (rw,noexec,nosuid,nodev,mode=0755)
varlock on /var/lock type tmpfs (rw,noexec,nosuid,nodev,mode=1777)
udev on /dev type tmpfs (rw,mode=0755)
devshm on /dev/shm type tmpfs (rw)
devpts on /dev/pts type devpts (rw,gid=5,mode=620)
compartida on /home/gr01/compartida type vboxsf (uid=1000,gid=1000,rw)
\end{envCodigo}

\item \texttt{df} ?`Qu'e espacio libre tiene cada uno de los filesystems montados?

\begin{envCodigo}
gr01@mondiola:~$ df -h
Filesystem            Size  Used Avail Use% Mounted on
/dev/sda1             3.0G  2.3G  471M  84% /
varrun                251M  100K  251M   1% /var/run
varlock               251M     0  251M   0% /var/lock
udev                  251M   40K  251M   1% /dev
devshm                251M     0  251M   0% /dev/shm
compartida             33G   30G  3.2G  91% /home/gr01/compartida
\end{envCodigo}

\item \texttt{ps} ?`Cu'antos procesos de usuario tiene ejecutando? Indique cu'antos son del sistema.

\begin{envCodigo}
gr01@mondiola:~$ ps
  PID TTY          TIME CMD
 5068 pts/0    00:00:02 bash
 5458 pts/0    00:00:00 man
 5468 pts/0    00:00:00 pager
26937 pts/0    00:00:00 ps
\end{envCodigo}

\begin{envCodigo}
gr01@mondiola:~$ ps aux
USER       PID %CPU %MEM    VSZ   RSS TTY      STAT START   TIME COMMAND
root         1  0.0  0.3   2844  1688 ?        Ss   Jun15   0:01 /sbin/init
root         2  0.0  0.0      0     0 ?        S<   Jun15   0:00 [kthreadd]
root         3  0.0  0.0      0     0 ?        S<   Jun15   0:00 [migration/]
root         4  0.0  0.0      0     0 ?        S<   Jun15   0:01 [ksoftirqd/]
root         5  0.0  0.0      0     0 ?        S<   Jun15   0:00 [watchdog/0]
root         6  0.0  0.0      0     0 ?        S<   Jun15   0:01 [events/0]
root         7  0.0  0.0      0     0 ?        S<   Jun15   0:00 [khelper]
root        36  0.0  0.0      0     0 ?        S<   Jun15   0:02 [kblockd/0]
root        39  0.0  0.0      0     0 ?        S<   Jun15   0:00 [kacpid]
root        40  0.0  0.0      0     0 ?        S<   Jun15   0:00 [kacpi_noti]
root        82  0.0  0.0      0     0 ?        S<   Jun15   0:00 [kseriod]
root       117  0.0  0.0      0     0 ?        S    Jun15   0:00 [pdflush]
root       118  0.0  0.0      0     0 ?        S    Jun15   0:02 [pdflush]
root       119  0.0  0.0      0     0 ?        S<   Jun15   0:00 [kswapd0]
root       162  0.0  0.0      0     0 ?        S<   Jun15   0:00 [aio/0]
root      1312  0.1  0.0      0     0 ?        S<   Jun15   0:04 [ata/0]
root      1315  0.0  0.0      0     0 ?        S<   Jun15   0:00 [ata_aux]
root      1320  0.0  0.0      0     0 ?        S<   Jun15   0:00 [ksuspend_u]
root      1327  0.0  0.0      0     0 ?        S<   Jun15   0:00 [khubd]
root      2452  0.0  0.1   2260   752 ?        S<s  Jun15   0:01 /sbin/udevd
root      2665  0.0  0.0      0     0 ?        S<   Jun15   0:00 [kpsmoused]
dhcp      3709  0.0  0.1   2436   552 ?        S<s  Jun15   0:00 dhclient3 -e
root      4094  0.0  0.0   1844   232 ?        Ss   Jun15   0:00 /usr/sbin/vb
root      4173  0.0  0.2   2456  1356 ?        Ss   Jun15   0:00 /usr/sbin/ac
root      4212  0.0  0.0      0     0 ?        S<   Jun15   0:00 [kondemand/]
syslog    4276  0.0  0.1   1936   684 ?        Ss   Jun15   0:00 /sbin/syslog
root      4414  0.0  0.1   5316   992 ?        Ss   Jun15   0:00 /usr/sbin/ss
avahi     4435  0.0  0.2   2756  1380 ?        Ss   Jun15   0:00 avahi-daemon
avahi     4436  0.0  0.0   2756   460 ?        Ss   Jun15   0:00 avahi-daemon
root      4743  4.8  3.3  24764 16972 tty7     Rs+  Jun15   3:13 /usr/bin/X :
root      4853  0.0  0.0   1716   508 tty1     Ss+  Jun15   0:00 /sbin/getty
gr01      4865  0.0  0.0   1772   504 ?        Ss   Jun15   0:00 /bin/sh /etc
gr01      4965  0.6  0.3   5256  1652 ?        Sl   Jun15   0:24 /usr/bin/VBo
gr01      4969  0.2  0.2   5256  1060 ?        Sl   Jun15   0:08 /usr/bin/VBo
gr01      4984  0.0  0.1   4480   536 ?        Ss   Jun15   0:00 /usr/bin/ssh
gr01      5023  0.0  0.9  15168  4968 ?        S    Jun15   0:00 Thunar --sm-
gr01      5026  0.0  2.6  34416 13820 ?        S    Jun15   0:03 xfdesktop --
gr01      5028  0.2  2.5  24340 13024 ?        S    Jun15   0:09 xfce4-panel
gr01      5068  0.0  0.8   6812  4276 pts/0    Ss   Jun15   0:02 bash
gr01      5338  0.0  0.8   7100  4536 pts/1    Ss+  Jun15   0:00 bash
gr01      5438  2.1  2.8  32220 14548 ?        Ss   Jun15   1:17 gvim tp1.tex
gr01      5458  0.0  0.2   3236  1104 pts/0    T    00:05   0:00 man find
gr01      5468  0.0  0.1   3236   932 pts/0    T    00:05   0:00 pager -s
gr01     26746  0.8  7.4  59924 38096 ?        Sl   00:27   0:16 evince tp1.p
gr01     26938  0.0  0.1   2644  1008 pts/0    R+   00:58   0:00 ps aux
\end{envCodigo}

\item \texttt{umount} Desmonte el CD-ROM de instalaci'on de Ubuntu JeOS.

\begin{envRespuesta}
gr01@mondiola:~\$ sudo umount /media/cdrom
\end{envRespuesta}

\item \texttt{uptime} ?`Cuanto tiempo lleva ejecutando su m'aquina virtual?

\begin{envCodigo}
gr01@mondiola:~$ uptime 
 01:05:38 up  1:14,  3 users,  load average: 0.21, 0.33, 0.35
\end{envCodigo}

\item \texttt{uname} ?`Qu'e versi'on del kernel de Linux est'a utilizando?

\begin{envCodigo}
gr01@mondiola:~$ uname -a
Linux mondiola 2.6.24-24-virtual #1 SMP Wed Apr 15 17:16:48 UTC 2009 i686 GNU/Linux
\end{envCodigo}

\end{enumerate}

\subsection{Salida est\'andar y pipes}

\begin{enumerate}

\item STDOUT

\begin{enumerate}

\item Conserve en el archivo \texttt{/home/<usuario>/tp/config} la salida del comando \texttt{ls} que muestra todos los
archivos del directorio \texttt{/etc} y de los subdirectorios bajo \texttt{/etc}.

\begin{envCodigo}
gr01@mondiola:~$ ls -R /etc/ > ~/tp/config
ls: cannot open directory /etc/cups/ssl: Permission denied
ls: cannot open directory /etc/ssl/private: Permission denied
\end{envCodigo}

\item Presente cuantas l'ineas, palabras y caracteres tiene \texttt{/home/<usuario>/tp/config}.

\fboxsep 0pt
\begin{envCodigo}
gr01@mondiola:~$ wc -l tp/config 
2553 tp/config
gr01@mondiola:~$ wc -w tp/config 
2295 tp/config
gr01@mondiola:~$ wc -m tp/config 
34633 tp/config
\end{envCodigo}

\item Agregue el contenido, ordenado alfab'eticamente, del archivo \texttt{/etc/passwd} al final del
archivo \texttt{/home/<usuario>/tp/config}.

\begin{envCodigo}
gr01@mondiola:~$ sort /etc/passwd >> ~/tp/config
\end{envCodigo}

\item Presente cuantas l'ineas, palabras y caracteres tiene \texttt{/home/<usuario>/tp/config}.

\begin{envCodigo}
gr01@mondiola:~$ wc -l tp/config 
2586 tp/config
gr01@mondiola:~$ wc -w tp/config 
2345 tp/config
gr01@mondiola:~$ wc -m tp/config 
36177 tp/config
\end{envCodigo}

\end{enumerate}

\item Pipes

\begin{enumerate}

\item Liste en forma amplia los archivos del directorio \texttt{/usr/bin} que comiencen con la letra ``a''.
Del resultado obtenido, seleccione las l'ineas que contienen el texto \texttt{apt} e informe la cantidad de caracteres,
palabras y l'ineas.

Est'a prohibido, en este 'item, usar archivos temporales de trabajo.

\begin{envCodigo}
gr01@mondiola:~/informeTP$ ls -lh /usr/bin/a* | grep apt
-rwxr-xr-x 1 root root  63K 2009-04-17 13:30 /usr/bin/apt-cache
-rwxr-xr-x 1 root root  18K 2009-04-17 13:30 /usr/bin/apt-cdrom
-rwxr-xr-x 1 root root 9.9K 2009-04-17 13:30 /usr/bin/apt-config
-rwxr-xr-x 1 root root  22K 2009-04-17 13:30 /usr/bin/apt-extracttemplates
-rwxr-xr-x 1 root root 187K 2009-04-17 13:30 /usr/bin/apt-ftparchive
-rwxr-xr-x 1 root root 139K 2009-04-17 13:30 /usr/bin/apt-get
-rwxr-xr-x 1 root root 2.2M 2008-04-04 06:56 /usr/bin/aptitude
-rwxr-xr-x 1 root root 1.9K 2008-04-04 06:56 /usr/bin/aptitude-create-state-bundle
-rwxr-xr-x 1 root root 3.0K 2008-04-04 06:56 /usr/bin/aptitude-run-state-bundle
-rwxr-xr-x 1 root root 5.0K 2009-04-17 13:30 /usr/bin/apt-key
-rwxr-xr-x 1 root root 2.2K 2009-04-17 13:30 /usr/bin/apt-mark
-rwxr-xr-x 1 root root  26K 2009-04-17 13:30 /usr/bin/apt-sortpkgs
-rwxr-xr-x 1 root root  12K 2008-04-15 18:40 /usr/bin/apturl
\end{envCodigo}

\end{enumerate}

\end{enumerate}

\subsection{Scripting}

Escriba un script de shell que sincronice dos carpetas. El script debe recibir las dos carpetas (origen y destino, en ese
orden) desde la l'inea de comando, o preguntarlas interactivamente al usuario en caso de no recibirlas. Adem'as, debe
aceptar un modificador \texttt{-r} que indica modo de operaci'on recursivo.

Una vez conocidas las dos carpetas con las que se operar'a, el script deber'a copiar todos los archivos de la carpeta origen
que no est'en presentes en la carpeta destino, y adem'as tambi'en deber'a copiar todos los archivos presentes en ambas
carpetas que tengan una fecha de modificaci'on posterior en la carpeta origen que en la carpeta destino. De esta manera, al
ejecutar el script, estar'a seguro de que la carpeta destino contiene toda la informaci'on de la carpeta origen, excepto lo
que fue modificado posteriormente en la carpeta destino.

El modificador \texttt{-r} indica al script realizar la sincronizaci'on tambi'en con los archivos de todos los
subdirectorios de la carpetas origen y destino.

\CodigoF{Script de sincronizaci'on}{grupo1/scriptSincro.sh}

\paragraph{Sugerencia}

Recuerde que generalmente el esp'iritu de los scripts es proveer la m'inima l'ogica necesaria alrededor de otros programas
ya presentes en el sistema que puedan proveer parte de la funcionalidad requerida para resolver un determinado problema.

\subsection{Ejecuci\'on de procesos en background}

Antes de resolver esta secci'on instale los siguientes paquetes en la m'aquina virtual:

\begin{itemize}
\item \texttt{nano}: editor de texto.
\item \texttt{mc}: manejador de archivos.
\item \texttt{gcc}: compilador de C.
\item \texttt{libc6-dev}: biblioteca est'andar de C.
\end{itemize}

Cree el archivo \texttt{/home/<usuario>/tp/loop.c}. Comp'ilelo con \texttt{gcc}. El programa compilado debe llamarse
\texttt{loop}.

\begin{envRespuesta}
gr01@mondiola:~/tp\$ gcc -o loop loop.c
\end{envRespuesta}

\CodigoF{loop.c}{grupo1/loop.c}

\begin{enumerate}
\item Correrlo en foreground. ?`Qu'e sucede? Mate el proceso con \texttt{Ctrl-c}.

\begin{envRespuesta}
gr01@mondiola:~/tp\$ ./loop\\
-120820995248484848484848484848484848484848484848484848484848\\
4848484848484848484848484848484848484848484848484848484848484\\
8484848484848484848484848484848484848484848484848484848484848\\
4848484848484848484848484848484848484848484848484848484848484\\
8484848484848484848484848484848484848484848484848484848484848\\
4848484848484848484848484848484848484848484848484848484848484\\
8484848484848484848484848484848484848484848484848484848484
\end{envRespuesta}

\item Ahora ejec'utelo en background: \texttt{/usr/src/loop > /dev/null \&}. Mate el proceso con el comando \texttt{kill}.

\begin{envCodigo}
gr01@mondiola:~/tp$ ./loop > /dev/null &
[1] 5467
gr01@mondiola:~/tp$ ps
  PID TTY          TIME CMD
 5084 pts/0    00:00:01 bash
 5467 pts/0    00:00:07 loop
 5510 pts/0    00:00:00 ps
gr01@mondiola:~/tp$ kill -9 5467 
gr01@mondiola:~/tp$ 
[1]+  Killed                  ./loop > /dev/null
gr01@mondiola:~/tp$ ps
  PID TTY          TIME CMD
 5084 pts/0    00:00:01 bash
 5961 pts/0    00:00:00 ps
\end{envCodigo}

\end{enumerate}

\subsection{IPC y sincronizaci\'on}

\subsubsection{Pipes}

Muestre con un ejemplo en lenguaje C como realizar exclusi'on mutua entre dos procesos utilizando un s'olo pipe.

\paragraph{Sugerencia}

Revise la ayuda de la llamada al sistema \texttt{pipe} para construir el pipe y de \texttt{fork} para crear
nuevos procesos.

\CodigoF{pipe.c}{grupo1/pipe.c}

\subsubsection{Threads}

Antes de resolver este ejercicio instale el paquete \texttt{glibc-doc}.

Resuelva el problema de productor/consumidor utilizando threads.

\paragraph{Sugerencia}

Revise la ayuda de \texttt{pthreads}, la implementaci'on de threads en Linux, para conocer los mecanismos de creaci'on y
destrucci'on de threads. Adem'as, \texttt{pthreads} provee mecanismos de sincronizaci'on que le ayudar'an a resolver este
ejercicio.


\CodigoF{productorConsumidor.c}{grupo1/productorConsumidor.c}

\begin{envRespuesta}
Para compilar: \\
gr01@mondiola:~/tp\$ gcc productorConsumidor.c -o productorConsumidor -lpthread
\\
gr01@mondiola:~/tp\$ ./productorConsumidor
\end{envRespuesta}

\subsection{El kernel Linux}

Antes de resolver esta secci'on instale los siguientes paquetes en la m'aquina
virtual:

\begin{itemize}
\item \texttt{make}: utilidad para mantener grupos de programas.
\item \texttt{linux-headers-<version>}: headers del kernel de Linux.
\end{itemize}

Sustituya \texttt{<version>} por el resultado del comando \texttt{uname -r}.

\begin{envRespuesta}
gr01@mondiola:~/tp\$ sudo apt-get install linux-headers-`uname -r`
\end{envRespuesta}

\subsubsection{Funcionamiento del kernel Linux}

\begin{enumerate}

\item Describa la administraci'on del procesador utilizada por defecto en el kernel Linux.
\vspace{.5cm}
PC o Process Scheduler (Programador de Procesos) es en realidad el
coraz'on del Sistema Operativo.

Sus responsabilidades son:

\begin{itemize}

\item Permitirle a los procesos crear nuevas copias de s'i mismos (forking)

\item Determinar qu'e proceso tendr'a acceso a la CPU

\item Recibir interrupciones y desviarlas hacia el subsistema respectivo

\item Enviar mensajes a los procesos de usuario

\item Manejar el reloj de Hardware (Hardware Timer)

\item Liberar recursos cuando un programa los haya desocupado

\end{itemize}

El Process Scheduler tambi'en soporta modulos cargados
din'amicamente. Los m'as importantes son el ``Virtual File System''
y ``Network Interface'' (explicados en puntos posteriores).

Tambi'en provee dos interfaces. Una, provee una interface de llamada
a sistema limitada que los procesos pueden llamar. Dos, provee una
interface para el resto del espacio del Kernel.

En tiempo de booteo, Linux posee s'olo un proceso, $init()$, que a
su vez realiza copias de s'i mismo a trav'es de llamadas $fork()$.
Cuando un proceso termina, ejecuta la llamada $a\_exit()$.

PS ademas posee llamados a se�ales a trav'es de $signal()$. Esta
llamada permite a un proceso asociar un manejador de funcion
(function handler) con una se�al en particular. Las se�ales pueden
considerarse como una forma de IPC o intercomunicaci'on de procesos.

Acerca del timer, Linux maneja un reloj interno que se inicia cada
10 milisegundos. Esto permite que los procesos se reasignen
(Scheduling) cada 10 ms. Esta unidad se llama ``clock tick'' y sirve
para medir la cantidad de ticks que un proceso en particular puede
continuar ejecut�ndose sin forzar su reasignaci'on.

Cada proceso se asigna con un identificador 'unico, llamado Process
ID o pid, que es asignado a la lista de procesos o tabla de
procesos.

Finalmente, una estructura modular es usada para representar los
modulos cargados en tiempo de ejecuci'on. Esta estructura posee
campos que son usados para implementar una lista de estructura de
modulos. Es decir, un campo que apunta a la tabla de s�mbolos de los
m'odulos y otro con su nombre.



\item Describa la administraci'on de memoria utilizada por defecto en el kernel Linux.
\vspace{.5cm}
El Memory Manager o administrador de memoria permite lo siguiente:

\begin{itemize}

\item Gran espacio de direcciones : Los programas pueden requerir m'as
memoria que la que f'isicamente existe

\item Protecci'on : La memoria asignada a un proceso es privada para tal

\item Mapas de Memoria : Se puede mapear un archivo dentro de un area de
memoria virtual y acceder al mismo como si fuera memoria
convencional

\item Acceso Limpio a la Memoria F'isica : el MM asegura que los procesos
puedan usar transparentemente todos los recursos de la m'aquina,
asegurando adem'as un rendimiento aceptable

\item Memoria Compartida : Permite que los procesos puedan compartir
trozos de la memoria asignada.

\end{itemize}

El Administrador de memoria posee dos interfaces : una interfaz con
llamada a sistema que es usada por los procesos de usuario (User
Space) y una interface que es usada por los otros subsistemas
(Kernel Space).

Algunas de las llamadas de usuario incluyen $malloc()$ y $free()$.
Las llamadas de kernel incluyen $kmalloc()$ y $kfree()$.

Ya que Linux soporta multiples arquitecturas, es necesario entonces
que existan rutinas espec'ificas para abstraer los detalles del uso
del hardware en una sola interface.

El Memory Manager usa el administrador de memoria de hardware para
mapear direcciones virtuales a direcciones f'isicas. Gracias a esto,
los procesos no est'an concientes de cuanta memoria f'isica est'a
asociada a una direcci'on virtual. Esto permite al MM poder mover
trozos de memoria virtual dentro de la memoria f'isica. Adem'as,
permite que dos procesos puedan compartir dos trozos de memoria
f'isica si las regiones de memoria virtual asignadas a ellas son
mapeadas en la misma direcci'on f'isica.

Otro concepto importante es el Swapping o intercambio, que es
intercambiar (swap) memoria ocupada por procesos en un archivo. Esto
le permite al kernel poder ejecutar mayor cantidad de procesos que
usen mayor cantidad de memoria f'isica existente. El MM posee un
m'odulo, $kswapd$, para ejecutar la tarea de intercambiar zonas de
memoria en archivos y viceversa. Este, chequea periodicamente si no
existen direcciones f'isicas mapeadas recientemente. Estas
direcciones son vaciadas de la memoria f'isica, y adem'as de 'esto,
el MM minimiza la cantidad de actividad de disco necesaria para
hacer este intercambio.

El administrador de memoria del hardware detecta cuando un proceso
de usuario tiene acceso a una porcion de memoria no mapeada en una
direcci�n virtual y notifica al kernel de esta falla. Existen dos
alternativas para solucionar esto: o la p'agina de memoria es
volcada a un archivo y viceversa; o el proceso est� haciendo
referencia a una zona de memoria no permitida.

Si el administrador de memoria detecta un acceso no permitido (a la
memoria) notifica al proceso con una se�al. Si el proceso no sabe
como manejar esta se�al, el proceso es finalizado.


\item Describa el sistema de archivos utilizado en la distribuci'on de Linux que instal'o en la m'aquina virtual.
?`Qu'e capas existen en el kernel Linux para soportar sistemas de
archivos sobre dispositivos de bloques?

\vspace{.5cm}

Linux est'a dise�ado para soportar distintos t'ipos de dispositivos
f'isicos. Incluso para un tipo espec'ifico, como un disco duro,
existen diferentes interfaces de manejo entre un fabricante de
hardware y otro. Adem'as de los dispositivos f'isicos, Linux posee
soporte para sistemas de archivos l'ogicos que lo hace interoperante
entre distintos sistemas operativos, con los siguientes prop'ositos:

\begin{itemize}

\item M'ultiples dispositivos f'isicos de hardware

\item M'ultiples sistemas de archivos l'ogicos

\item M'ultiples tipos de archivos ejecutables

\item Homogeneidad, es decir, una interfaz com'un entre los sistemas de
archivos l'ogicos y el hardware

\item Rendimiento

\item Seguridad de datos, perder o corromper datos

\item Seguridad de acceso, restricci'on a los archivos, quotas,
permisos, etc.

\end{itemize}

Tambi'en se usan dos interfaces para el manejo de filesystems. Una
interfaz para llamadas de usuarios y otra para los subsistemas del
kernel. La interfaz para los usuarios maneja archivos y directorios.
Las operaciones en archivos incluyen $open, read, close, write,
seek$, que est'an definidas en el estandar $POSIX$ y en los
directorios $readdir, creat, unlink, chmod, stat$ tambi'en definidas
en $POSIX$.

Pero la interface de los subsistemas del Kernel es bastante m'as
interesante. Esta posee estructuras de datos y funciones para
manipulaci'on directa para otros subsistemas. De hecho, existen dos
interfaces para el resto del kernel: inodos y archivos.

La interfaz de inodos posee soporte para $create()$, $lookup()$,
$link()/symlink()$, $mkdir()$, $mknod()$,etc.

La interfaz de archivos posee soporte para $open()/release()$,
$read()/write()$, $select()$, $mmap()$, $fsync()$, etc.

Volviendo con los controladores de dispositivos o drivers, Linux
posee tres tipos de drivers: $char$, $block$ y $network$. Los dos
m'as relevantes son $char$ y $block$. Los dispositivos $char$ son
aquellos que su lectura es secuencial (de a un char a la vez), como
un mouse. Los dispositivos $block$ pueden ser accedidos de cualquier
manera, pero solo puede leerse o escribirse de a bloques.

Todos los drivers soportan las operaciones mencionadas reci'en.
Adem'as, cada dispositivo puede ser accesado como si fuera un
archivo. Como el Kernel se entiende con los dispositivos de esa
manera, no es complicado agregar un nuevo dispositivo, ya que solo
es necesario implementar el c�digo espec'ifico del hardware para el
soporte de la abstracci'on de archivos.

Adem'as, el kernel posee un buffer de cach'e para mejorar el
rendimiento cuando usa dispositivos $block$. Todo acceso a un
dispositivo $block$ pasa a trav'es de un subsistema de buffer. Este
buffer aumenta considerablemente el rendimiento, minimizando las
lecturas y escrituras hacia y desde los dispositivos.

Cada dispositivo posee una cola de petici'on (request queue). Cuando
el buffer no puede responder una petici'on, se agrega una de 'estas
a esta cola y hace que la petici'on duerma (sleep) hasta que pueda
responder. Este buffer usa un espacio separado del kernel como un
thread 'unico, manejado por $kflushd$.

Existen distintos mecanismos para mover datos entre el computador
hacia los dispositivos
\begin{itemize}
\item Polling
\item DMA (acceso directo a memoria)
\item Interrupciones
\end{itemize}

En el caso de polling, el driver verifica peri'odicamente el CSR
(Control and Status Register) para ver si la petici'on ha sido
completada. Si es as'i, el dispositivo inicia la siguiente petici'on
y continua. Polling es altamente efectivo en dispositivos de
transferencia lenta, como disketteras y modems.

En el caso de DMA, el driver inicia una transferencia directa entre
la memoria del computador y el perif'erico. Esta transferencia es
concurrente con la CPU y permite que la CPU siga haciendo otras
tareas mientras se transfieren datos. Cuando la operaci'on termina,
la CPU recibe una interrupci'on.

Cuando un dispositivo desea cambiar de estado (por ejemplo,
presionar el bot'on del mouse) o reporta el final de una operaci'on,
env'ia una interrupci'on al procesador. Si las interrupciones estan
habilitadas, el procesador detiene su ejecuci'on actual y comienza a
ejecutar el c'odigo de manejo de interrupciones del kernel. El
kernel encuentra el driver correcto para invocar. Cuando una
interrupci'on es manejada, la CPU se ejecuta en un contexto
especial, es decir, todas las interrupciones quedan pendientes hasta
que la interrupci'on actual haya terminado. Debido a esto, los
manejadores de interrupciones deben ser altamente eficientes. Si
ocurre el caso que un manejador de interrupciones no puede terminar
el trabajo, reasigna el trabajo pendiente en un manejador de tipo
``bottom\-half'' , es decir, c'odigo que se ejecutar'a la pr'oxima
vez que la llamada sea completada para evitar latencias y promover
la concurrencia.

Volviendo ahora a los sistemas de archivos l'ogicos, es m'as f'acil
acceder a un dispositivo si es manejado como un archivo a que si
fuera manejado directamente en el hardware. Un filesystem l'ogico
puede ser montado en un punto de montaje dentro del filesystem
virtual. Esto significa que un bloque asociado a un dispositivo
contiene archivos y una informaci'on de estructura que permite al
filesystem l'ogico accederlo. Un dispositivo s'olo puede tener un
soporte l'ogico de archivo. Sin embargo, puede ser ``adiestrado''
para poseer soporte para un soporte l'ogico en otro tipo de
filesystem.

En pocas palabras y con una explicaci'on m'as clara, los
dispositivos por lo general se encuentran en $/dev$ y cada uno se
representa como un archivo. Es perfectamente posible reescribir su
driver para que soporte otro tipo de filesystem l'ogico.

Para el soporte de filesystems virtuales, Linux usa Inodos para
representar un archivo en un dispositivo block. El inodo es virtual
en el sentido que contiene operaciones que est'an implementadas de
distintas maneras, dependiendo del sistema l'ogico y f'isico donde
el archivo reside. El inodo es usado como un lugar de almacenamiento
para toda la informaci'on relacionada con el manejo de, por ejemplo,
abrir un archivo del disco. Guarda buffers, el largo del archivo ,
etc.

Mucha de la funcionabilidad de los filesystems virtuales radica en
los m'odulos. Esta configuraci'on permite a los usuarios compilar un
Kernel tan peque�o como sea posible, cargando solamente los modulos
que sean necesarios.


Ideal es el caso en el cual el dispositivo pudiera ocuparse y
liberarse en tiempo de ejecuci'on. Si el m'odulo fuera agreagado
activamente al kernel, se deben pasar sus par'ametros en momento de
booteo, sin considerar tampoco el gasto de memoria en mantener el
uso de un dispositivo que no est'e conectado, como por ejemplo, una
impresora.


\end{enumerate}

\subsubsection{M'odulos de kernel}

El kernel de Linux permite ser ampliado en \textbf{runtime} con m'odulos. Los m'odulos son objetos de c'odigo compilado
que pueden ser insertados en runtime al kernel, siendo linkeados contra el kernel al momento de ser insertados. De esta
manera puede ampliarser la funcionalidad del kernel en runtime, sin tener que incluir todo el c'odigo en el binario
original.

\subsubsection{Compilando un m'odulo de kernel}

\begin{envRespuesta}
En esta secci'on omitimos el enunciado y vamos directo a lo que hicimos. \\
\textbf{Nota:} Para ejecutar la siguiente parte debe pasar a modo consola (Host+F1).
\end{envRespuesta}

\CodigoF{cambiarLuces.c}{grupo1/cambiarLuces.c}

Para compilar este c'odigo deber'a construir una \textbf{Makefile}. Este archivo Makefile es utilizado luego por el
comando \texttt{make} para compilar el m'odulo con las opciones correctas. En el mismo directorio donde se encuentra
\texttt{cambiarLuces.c} cree un archivo \texttt{Makefile} conteniendo:

\CodigoF{Makefile}{grupo1/Makefile.txt}

Luego ejecute:

\texttt{make}

Para compilar el m'odulo. Finalmente, pruebe insertar el m'odulo usando:

\texttt{insmod cambiarLuces.ko} esto tambi'en viene facilitado con el siguiente script
\CodigoF{./cargarKernel.sh}{grupo1/cargarKernel.sh}

Deber'ia ver un mensaje en la consola indicando que se cargo correctamente o que fall'o.

Para hacer funcionar el m'odulo escribir como root en \texttt{/proc/lasluces/escribaAqui} o, nuevamente, con un script que toma un dato:

\CodigoF{./escribir.sh dato}{grupo1/escribir.sh}

Para sacar el m'odulo:

\texttt{rmmod cambiarLuces.ko} o su versi'on script.

\CodigoF{./descargarKernel.sh }{grupo1/descargarKernel.sh}

Esto tambi'en deber'ia dar un mensaje indicando lo sucedido.

\subsubsection{Un m'odulo propio}

Escriba un m'odulo de kernel que permita controlar los LEDs del teclado \textbf{sin necesidad de escribir un programa
en lenguaje C}. El m'odulo deber'a permitir prender o apagar cada LED usando simplemente comandos del shell.

Dado que la m'aquina virtual provista por VirtualBox no muestra el estado de los LEDs, la c'atedra provee abajo
un programa que lo hace. Compile utilizando:

\texttt{gcc -o check\_kbleds check\_kbleds.c}

\CodigoF{check\_kbleds.c}{grupo1/checkkbleds.c}

\paragraph{Sugerencia}

Haga que su m'odulo cree un archivo en \texttt{/proc} y que las escrituras a ese archivo controlen los LEDs utilizando
la IOCTL \texttt{KDSETLED} de la consola de Linux.

\subsection{Temas del sistema operativo}

\subsubsection{File system}
En el ejercicio 3.1.12 se hace menci'on a los hardlinks como apuntadores a un
mismo espacio de disco. C'omo se llama ese espacio de disco, que estructura
tiene y por qu'e se pueden borrar los hardlinks sin borrar al archivo.

\subsubsection{Prioridades}
Genere tres versiones del programa loop (3.4.1) (loop1, loop2 y loop3) y
ejec'utelos en background. Logre que loop3 ejecute más r'apido que los otros
dos (prioridad). Obtenga los tiempos de ejecuci'on de cada uno de ellos (uso de
procesador) y sus estados. Explique detalladamente como logra obtener esta
información.

\subsubsection{Par'ametros del Kernel}
Determine la cantidad de memoria RAM que usa su sistema, ahora encuentre una
manera para que use menor cantidad de memoria RAM. Explique detalladamente como
logra obtener este cambio y como obtiene la información.

\subsubsection{Administraci'on de Memoria}
Determine el tamaño de la partición swap que est'a utilizando. Ampl'ie el
tama~no del swap por medio de un archivo en un FS. H'agalo persistente.
Explique detalladamente como logra obtener este cambio.

\subsubsection*{(The End...)}

\begin{envRespuesta}
This is the end, beautiful friend\\
This is the end, my only friend\\
The end of our elaborate plans\\
The end of everything that stands\\
The end\\
\end{envRespuesta}


