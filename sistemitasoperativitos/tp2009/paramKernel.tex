Para conseguir una mayor cantidad de memoria a partir de par'ametros
pasados al Kernel, consultamos a la documentaci'on del
mismo(kernel$-$parameters.txt). En ella, nos explicaba que deb'iamos
usar la sentencia $mem=nn[K,M,G]$.

A partir de algunas pruebas, conseguimos el objetivo a partir de una
sentencia similar $mem=nn[K,M,G]@ss[K,M,G]$, que act'ua de la misma
manera que la anterior, pero $ss$ es el desplazamiento a partir del
cu'al se cargar'a el kernel.

\begin{envCodigo}
Carga normal del S.O.
gr01@mondiola:~$ free
             total       used       free     shared    buffers     cached
Mem:        513624     246860     266764          0       8892     105228
-/+ buffers/cache:     132740     380884
Swap:       201788          0     201788


Agregando a la linea del kernel en el grub mem=510M@64M
gr01@mondiola:~$ free
             total       used       free     shared    buffers     cached
Mem:        513624     246740     266884          0       8860     105104
-/+ buffers/cache:     132776     380848
Swap:       201788          0     201788
\end{envCodigo}
