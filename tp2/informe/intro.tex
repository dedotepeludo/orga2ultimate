\section{Introducci'on}

El objetivo de este tp, es optimizar las funciones que hab'iamos realizado
durante el desarrollo del tp1 utilizando, esta vez, la tecnolog'ia \textbf{MMX}
para poder, de esta forma, ``agilizar'' nuestro programa. Adem'as, se agreg'o
una funci'on que nos permitiera utilizar la Unidad de Punto Flotante
(\textbf{FPU}) para realizar c'alculos con n'umeros no enteros. 

Una breve descripci'on de las funciones que realizamos, a modo de
introducci'on, ser'ia la siguiente:

\begin{itemize}
	\item \textbf{generarFondo}: Pinta el cielo y copia el piso.
	\item \textbf{recortar}: Recorta una parte de un sprite.
	\item \textbf{blit}: Saca el color de off para mimetizar la imagen con el fondo.
	\item \textbf{chequearColisiones}: Comprueba si hay colisi'on entre 2 objetos, en este
caso, \textbf{Mario} y el ca~no.
	\item \textbf{generarRayo}: Es lo que produce que el rayo de \textbf{Mario}
se vea.
	\item \textbf{apagar}: Va cambiando el color de las monedas.
\end{itemize}

Esperemos que distrute la lectura nuestro trabajo.

