\section{Manual del usuario}

\subsection{Detalles del contenido del cd}

	En el \textbf{CD} usted podr'a ver 6 archivos .asm. Cada uno se llama como
la funci'on cuyo codigo guarda en su interior, un archivo Makefile para poder
compilar el tp, un archivo .c y un directorio im'agenes con su contenido, estos
'ultimos dos provistos por la c'atedra.

	El listado de archivos ser'ia:
\begin{verbatim}
		apagar.asm     
		chequearColisiones.asm  
		generarRayo.asm 
		blit.asm 
		imagenes 
		generarFondo.asm
		recortar.asm
		Makefile
		main.c
		tp2.pdf

		./imagenes:
			bs.bmp
			coins.bmp
			hspt.bmp
			marioJumpL.bmp
			marioStandL.bmp
			pipe.bmp
			questB.bmp
			cloud.bmp
			estupe.bmp
			hsptr.bmp
			marioSpriteW.bmp
			marioStandR.bmp
			piso.bmp
			quest.bmp
			coin.bmp
			go.bmp
			marioJump.bmp
			marioSpriteWL.bmp
			mon.bmp
			PrincessToadstool.bmp
			victoria.bmp
\end{verbatim}

\subsection{Uso del contenido del cd}

El cd viene con sus correpondiente Makefile para correr el tp como ya
explicamos en la entrega del tp1.

Para ganar el juego usted debe agarrar un par de monedas con el rayo, 
agarrar una vez la caja y despu'es parado lo m'as a la derecha que 
se pueda dispararle al ca~no.

Aunque cabe aclarar que resulta m'as divertido o pintoresco o raro o 
sorprendente o innovador o inesperado o jeje o asombroso o increible o 
intrigante o intr'inseco o divagante o extravagante o extra~no o amazing
o ... o groso perder el juego cosa que se logra disparandole al ca~no o 
a muchas monedas.
