\section{Conclusiones}
A lo largo del trabajo pudimos entender en profundidad el concepto de
little-endian. Al tener que trabajar con tres bytes utilizando registros de
cuatro nos ha dado varios dolores de cabeza a la hora de leer y escribir
p'ixeles. Creemos que este fue uno de los obst'aculos mas importantes del
trabajo pr'actico.

Para nuestra sorpresa, pudimos ver que pasar pseudoc'odigo al
\textit{assembler} se puede utilizar un orden similar al de lenguajes de mas
alto nivel, pero sin perder de vista las restricciones que ofrece assembler
(cantidad de registros, operaciones, etc).

Inicialmente generamos todas las funciones vacias y a medida en que fuimos
implement'andolas se noto una baja de performance considerable. Esto se veia
claramente en la velocidad en la que camina \textbf{Mario} a lo largo de la pantalla y
en especial en las m'aquinas del laboratorio 5 del DC (\textit{Labo Linux})
siendo 'estas ``m'aquinas de 'ultima generaci'on''.

Tambi'en pudimos ver que no podemos fiarnos de los resultado gr'aficos hasta
que no implementamos 'integramente todo el tp. Por ejemplo, al terminar
\textbf{checkColision} cuando \textbf{Mario} colisionaba con el pipe se sub'ia a 'este y luego
segu'ia caminando en el aire. Esto nos hizo pensar mucho tiempo que el
problema estaba en la implementaci'on de \textbf{checkColision} hasta que decidimos
seguir e implementar \textbf{salta} despu'es de lo cual nos alegramos
enormemente al ver que \textbf{Mario} hac'ia lo que ten'ia que hacer. As'i que con la
satisfacci'on del deber cumplido nos tomamos una birra para festejar.

Teniendo en cuenta los problemas que han tenido algunos de nuestros compa~neros
en windows podemos concluir que trabajar bajo la plataforma de linux nos ha
ahorrado muchos de estos problemas y ha sido la mejor opci'on de trabajo.

