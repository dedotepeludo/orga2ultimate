\section{Conclusiones}
A lo largo del trabajo pudimos entender en profundidad el concepto de little-endian. Al tener que trabajar con tres bytes utilizando registros de cuatro nos ha dado varios dolores de cabeza a la hora de leer y escribir p'ixeles. 
Creemos que este es uno de los obst'aculos mas importantes del trabajo pr'actico.

Pudimos ver que pasar pseudoc'odigo al assembler se puede utilizar un orden similar al de lenguajes de mas alto nivel, pero sin perder de vista las restricciones que ofrece assembler (cantidad de registros, operaciones, etc).

Inicialmente generamos todas las funciones vacias y a medida en que fuimos implementandolas se noto una baja de performance considerable. Esto se veia claramente en la velocidad en la que camina Mario a lo largo de la pantalla y en especial en las m'aquinas del laboratorio de linux siendo estas ``m'aquinas de ultima generaci'on''.

Tambi'en pudimos ver que no podemos fiarnos de los resultado gr'aficos hasta que no implementamos integramente todo el tp. Por ejemplo, al terminar checkColision cuando Mario colisionaba con el pipe se subia a este y luego seguia caminando en el aire. Esto nos hizo pensar mucho tiempo creyendo que el problema estaba en la implementaci'on de checkColision hasta que decidimos seguir e implementar salta. Al finalizar nos alegramos mucho al ver que Mario hacia lo que tenia que hacer y nos tomamos una birra para festejar.

Teniendo en cuenta los problemas que han tenido algunos de nuestros compa~neros en windows podemos concluir que trabajar bajo la plataforma de linux nos ha ahorrado varios problemas y ha sido la mejor opci'on de trabajo.

