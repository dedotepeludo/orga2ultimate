\section{Manual del usuario}

\subsection{Detalles del contenido del cd}

	En el \textbf{CD} usted podr'a ver 6 archivos .asm cada uno se llama como
la funci'on cuyo codigo guarda en su interior, un archivo Makefile para poder
compilar el tp, un archivo .c y un directorio imagenes con su contenido, estos
'ultimos 2 provistos por la c'atedra.

	El listado de archivos ser'ia:
\begin{verbatim}
		apagar.asm
		blit.asm
		checkcolision.asm
		generarFondo.asm
		main.c
		Makefile
		recortar.asm
		salta.asm

		imagenes:
			cloud.bmp
			coin.bmp
			marioJump.bmp
			marioJumpL.bmp
			marioSpriteW.bmp
			marioSpriteWL.bmp
			marioStand.bmp
			mon.bmp
			pipe.bmp
			piso.bmp
			PrincessToadstool.bmp
			questB.bmp
			quest.bmp
			victoria.bmp
\end{verbatim}

\subsection{Uso del contenido del cd}
	Dentro del material entregado en el medio extraible digital que adjuntamos
a este informe, cuenta usted con un \textbf{Makefile}. >Qu'e quiere decir eso? Quiere
decir que si ud., en la misma plataforma en la que nosotros trabajamos, escribe
la palabra ``make'' (en el directorio donde est'a el archivo \textbf{Makefile})
y presiona enter ($\dlsh$), tendr'a el TP compilado y listo para probar.
Aseg'urese de tener instalado el \textbf{nasm}, el \textbf{gcc} y la
\textbf{libsdl-gfx1.2-dev} para que esta acci'on se realice con 'exito. Una vez
que ya hizo lo antes mencionado, podr'iamos decir que usted ya tiene compiladas
y linkeadas todas las partes del tp y generado el ejecutable que le permitir'a
correr el programa generado por nuestro c'odigo. Se generan 6 archivos~.o
correspondientes a los 6 archivos~.asm (uno por cada funci'on que tuvimos que
realizar) y el archivo ejecutable que nos permite correr el programa. Como
corresponde, este archivo se genera bajo el nombre de ``runner''.

	El \textbf{Makefile} permite adem'as correlo con el par'ametro ``clean'',
es decir ``make clean'' que borra todos los archivos generados al hacer el
``make''.

	Para correr el programa, simplemente tipee ``./runner'' y presione enter.
Una ventana se abrir'a y podr'a usted pasar un buen momento haciendo saltar,
mover, trepar y golpear cubitos de monedas a \textbf{Mario}.

	Si bien estamos en contra de que en el manual del usuario se expliquem los
pasos a dar en el mismo para ganarlo, creemos que, dado los tiempos veloces
que corren en el siglo XXI, quiz'as no le alcance el tiempo al corrector para
descubrirlo o para poder preguntarnos a nosotros, por lo cual le comentamos (no
lea de aqu'i en m'as si no quiere saber la soluci'on), que la soluci'on...
>est'a seguro de que quiere saberla y no descubrirla usted mismo?... es saltar
muchas veces sobre el ca~no. Esta acci'on tiene un efecto extra~no en
\textbf{Mario} que
lo hace saltar cada vez m'as alto, como si arriba del ca~no estuviera haciendo
una dieta del Doctor Cormillot y a cada salto fuera m'as y m'as liviano. Este
efecto se pierde y todo vuelve a la normalidad cuando \textbf{Mario} vuelve a pisar
tierra firme o deja de saltar. Si \textbf{Mario} llega hasta el cielo, o parte superior
de la pantalla, y colisiona con este, \textbf{Mario} ``libera'' a la
\textbf{Princesa}. La \textbf{Princesa Durazno} aparece en la pantalla y el
juego termina cuando \textbf{Mario} llega a saludarla.

	Eso es todo lo que podemos informarle sobre nuestra versi'on del juego.
Dudas, quejas, sugerencias a: grupo\_orga\_2@googlegroups.com . Si esta
direcci'on no funciona le pedimos disculpas y le rogamos que reenv'ie su mail a
las direcciones de mail que figuran en la tapa de este informe. 

	Desde ya muchas gracias y esperemos que este juego e informe le hayan
resultado amenos de leer y entretenido.

	Plantel del grupo:
\begin{itemize}
 \item Cristian
 \item Mat'ias
 \item Tommy
\end{itemize}

	<Hasta la pr'oxima!
