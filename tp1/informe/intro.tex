\section{Introducci'on}

	Para la realizaci'on de este trabajo pr'actico nos fue necesario, entre
otras cuestiones log'isticas, aprender a lidiar con las instrucciones de \textit{assembler}.
Esta tarea no nos result'o sencilla en un primer momento, pero gracias a que
mientras hac'iamos tp, tambi'en estudiabamos para el parcial, fuimos aprendiendo
rapidamente.

	Uno de los puntos m'as importantes que cabe destacar en la
realizaci'on de este trabajo pr'actico es la gran comodidad que nos permiti'o
para realizarlo la posibilidad de contar con un \textit{svn} para no tener que perder
tiempo intercambiandonos la 'ultima versi'on y buscando versiones viejas. Esto
nos result'o muy c'omodo ni bien empezamos a programar, cuando de golpe tras
cambiar 3 o 4 lineas el programa daba Segmentation Fault (``\textit{Segfaulteaba}'').

	Como metodolog'ia de trabajo no utilizamos ninguna en particular m'as que
ir haciendo las funciones en el orden que figuraban en el enunciado. Esta
``decisi'on'' se ve reflejada en la calidad del c'odigo. Las 'ultimas funciones
aprovechan mucho m'as las operaciones del lenguaje. Desgraciadamente no
contamos con tiempo como para rehacer el trabajo pr'actico, ya que sin duda si
encararamos nuevamente las funciones nos saldr'ian distintas. Esto refleja
nuestro aprendizaje a lo largo del desarrollo del tp.

	Salvo las funciones \textbf{checkcolision} y \textbf{salta}, las dem'as
(\textbf{apagar}, \textbf{generarFondo}, \textbf{blit}, \textbf{recortar}) son
funciones que modifican directamente la estructura de un p'ixel. Las 2 primeras
son puramente num'ericas que realizan algunos c'alculos y comparaciones.

	Brevemente podemos describir el funcionamiento de estas 6 funciones as'i
(en orden de realizaci'on):

\begin{itemize}
	\item \textbf{generarFondo}: Pinta el cielo y copia el piso.
	\item \textbf{recortar}: Recorta una parte de un sprite.
	\item \textbf{blit}: Saca el color de off para mimetizar la imagen con el fondo.
	\item \textbf{checkcolision}: Comprueba si hay colisi'on entre 2 objetos, en este
caso, \textbf{Mario} y el ca~no o la caja de las monedas.
	\item \textbf{salta}: Maneja el comportamiento de \textbf{Mario} cuando salta.
	\item \textbf{apagar}: Va cambiando el color de las monedas.
\end{itemize}

